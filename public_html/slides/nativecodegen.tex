\documentclass[portrait]{seminar}
%\usepackage{pandora}
\usepackage{color}
\usepackage{fancybox}
\usepackage{alltt}
\usepackage{epsfig}
\usepackage{rail}
\usepackage{bar}
\usepackage{url}
\usepackage{rotating}
\usepackage[normalem]{ulem}
\usepackage{latexsym}
\usepackage{amsmath}

\begin{document}

\boldmath
\newcommand{\RA}{$\rightarrow$}
\newcommand{\LL}{\mbox{$[$\hspace{-0.15em}$[$}}
\newcommand{\RR}{\mbox{$]$\hspace{-0.15em}$]$}}
\newcommand{\CC}[1]{\mbox{\tt $\LL$#1$\RR$}}

\slideframe{shadow}

%%% Activate one of these to get either Aarhus style or McGill style 
%%% by putting a #1 in the appropriate line.
\newcommand{\mcgill}[1]{#1}
\newcommand{\aarhus}[1]{}

%%% Define this to be the name of your term
\newcommand{\courseterm}{Fall 2012}




\aarhus{
\newpagestyle{dOvsstyle}{dOvs'98 Week 48 \hfil Native code generation}{\hfil \thepage}
}

\mcgill{
\newpagestyle{dOvsstyle}{COMP 520 \courseterm  \hfil Native code generation (\thepage)}{}
}
\slidepagestyle{dOvsstyle}

\begin{slide*}
\begin{tabbing}
\aarhus{{\Large\bf Week 48}\\}
~\\
{\Huge\bf Native code}\\ ~\\ {\Huge\bf generation}\\
\end{tabbing}
\vfil
\end{slide*}

\begin{slide*}
JOOS programs are compiled into bytecode.\\

This bytecode can be executed thanks to either:

\begin{itemize}
\item an interpreter;
\item an Ahead-Of-Time (AOT) compiler; or
\item a Just-In-Time (JIT) compiler.
\end{itemize}
\vspace*{1em}

Regardless, bytecode must be implicitly or explicitly translated into
native code suitable for the host architecture before execution.
\vfil
\end{slide*}
 
\begin{slide*}
Interpreters:

\begin{itemize}
\item are easier to implement;
\item can be very portable; but
\item suffer an inherent inefficiency:
\end{itemize}

\begin{tt}
\begin{scriptsize}
\begin{verbatim}

pc = code.start;
while(true)
  {  npc = pc + instruction_length(code[pc]);
     switch (opcode(code[pc]))
       {  case ILOAD_1: push(local[1]);
                        break;
          case ILOAD:   push(local[code[pc+1]]);
                        break;
          case ISTORE:  t = pop();
                        local[code[pc+1]] = t;
                        break;
          case IADD:    t1 = pop();  t2 = pop();
                        push(t1 + t2);
                        break;
          case IFEQ:    t = pop();
                        if (t == 0) npc = code[pc+1];
                        break;
          ...
       }
     pc = npc;
  }
\end{verbatim}
\end{scriptsize}
\end{tt}
\vfil
\end{slide*}
 
\begin{slide*}
Ahead-of-Time compilers:

\begin{itemize}
\item translate the low-level intermediate form into native code;
\item create all object files, which are then linked, and finally executed.
\end{itemize}
\vspace*{1em}

This is not so useful for Java and JOOS:

\begin{itemize}
\item method code is fetched as it is needed;
\item from across the internet; and
\item from multiple hosts with different native code sets.
\end{itemize}
\vfil
\end{slide*}
 
\begin{slide*}
Just-in-Time compilers:

\begin{itemize}
\item merge interpreting with traditional compilation;
\item have the overall structure of an interpreter; but
\item method code is handled differently.
\end{itemize}
\vspace*{1em}

When a method is invoked for the first time:
\begin{itemize}
\item the bytecode is fetched;
\item it is translated into native code; and
\item control is given to the newly generated native code.
\end{itemize}
When a method is invoked subsequently:
\begin{itemize}
\item control is simply given to the previously generated native code.
\end{itemize}

\vfil
\end{slide*}
 
\begin{slide*}
Features of a JIT compiler:\\
\begin{itemize}
\item it must be {\em fast}, because the compilation occurs
      at run-time (Just-In-Time is really Just-Too-Late);
\item it does not generate optimized code;
\item it does not compile every instruction into native code, but relies on the
runtime library for complex instructions;
\item it need not compile every method; and
\item it may concurrently interpret and compile a method (Better-Late-Than-Never).
\end{itemize}
\vfil
\end{slide*}
 
\begin{slide*}
Problems in generating native code:\\

\begin{itemize}
\item {\em instruction selection}:\\ choose the correct instructions based
                                      on the native code instruction set;
\item {\em memory modelling}:\\ 
                                   decide where to store variables and how to allocate registers;
\item {\em method calling}:\\ determine calling conventions; and
\item {\em branch handling}:\\ allocate branch targets.
\end{itemize}
\vfil
\end{slide*}
 
\begin{slide*}
Compiling JVM bytecode into VirtualRISC:\\

\begin{itemize}
\item map the Java local stack into registers and memory;
\item do instruction selection on the fly;
\item allocate registers on the fly; and
\item allocate branch targets on the fly.
\end{itemize}
\vspace*{2em}

This is successfully done in the Kaffe system.
\vfil
\end{slide*}

\begin{slide*}
The general algorithm:\\

\begin{itemize}
\item determine number of slots in frame:\\
      locals limit + stack limit + \#temps;
\item find starts of basic blocks;
\item find local stack height for each bytecode;
\item emit prologue;
\item emit native code for each bytecode; and
\item fix up branches.
\end{itemize}
\vfil
\end{slide*}

\begin{slide*}
Na\"Ive approach:\\

\begin{itemize}
\item each local and stack location is mapped to an offset in the native frame; 
\item each bytecode is translated into a series of native instructions, which
\item constantly move locations between memory and registers.
\end{itemize}
\vspace*{2em}

This is similar to the native code generated by a non-optimizing compiler.
\vfil
\end{slide*}

\begin{slide*}
Example:

\begin{scriptsize}
\begin{verbatim}
public void foo() {
  int a,b,c;
 
  a = 1;
  b = 13;
  c = a + b;
}
\end{verbatim}
\end{scriptsize}

Generated bytecode:

\begin{scriptsize}
\begin{verbatim}
.method public foo()V
  .limit locals 4
  .limit stack 2
  iconst_1         ; 1
  istore_1         ; 0
  ldc 13           ; 1
  istore_2         ; 0
  iload_1          ; 1
  iload_2          ; 2
  iadd             ; 1
  istore_3         ; 0
  return           ; 0
\end{verbatim}
\end{scriptsize}

\begin{itemize}
\item compute frame size = 4 + 2 + 0 = 6;
\item find stack height for each bytecode;
\item emit prologue; and
\item emit native code for each bytecode.
\end{itemize}
\vfil
\end{slide*}

\begin{slide*}
\renewcommand{\arraystretch}{0.9}
Assignment of frame slots:
\vspace*{0.5ex}

\begin{center}
\begin{scriptsize}
\begin{tt}
\begin{tabular}{l@{~~~}l@{~~~}l}
name & offset & location \\ \\
a & 1 & [fp-32] \\
                 b    & 2  &  [fp-36]\\
                 c    & 3  &  [fp-40]\\
 \\
               stack  & 0  &  [fp-44]\\
               stack  & 1  &  [fp-48]
\end{tabular}
\end{tt}
\end{scriptsize}
\end{center}
\vspace*{0.5ex}

Native code generation:

\begin{center}
\begin{scriptsize}
\begin{tt}
\begin{tabular}{l@{~~~}l@{~~~}l}
        &                    &    save sp,-136,sp\\
a = 1;             &     iconst\_1       &    mov 1,R1\\
                   &                    &    st  R1,[fp-44]\\
                   &     istore\_1       &    ld [fp-44],R1\\
                   &                    &    st R1,[fp-32]\\
b = 13;            &     ldc 13      &    mov 13, R1\\
                   &                    &    st R1,[fp-44]\\
                   &     istore\_2       &    ld [fp-44], R1\\
                   &                     &   st R1,[fp-36]\\
c = a + b;         &     iload\_1        &    ld [fp-32],R1\\
                   &                    &    st R1,[fp-44]\\
                   &     iload\_2        &    ld [fp-36],R1\\
                   &                    &    st R1,[fp-48]\\
                   &     iadd           &    ld [fp-48],R1\\
                   &                    &    ld [fp-44],R2\\
                   &                    &    add R2,R1,R1\\
                   &                    &    st R1,[fp-44]\\
                   &     istore\_3       &    ld [fp-44],R1\\
                   &                    &    st R1,[fp-40]\\
                   &     return         &    restore\\
                   &                    &    ret
\end{tabular}
\end{tt}
\end{scriptsize}
\end{center}
\vfil
\renewcommand{\arraystretch}{1}
\end{slide*}
 
\begin{slide*}
\renewcommand{\arraystretch}{0.9}
The na\"ive code is very slow:

\begin{itemize}
\item many unnecessary loads and stores, which
\item are the {\em most\/} expensive operations.
\end{itemize}
\vspace*{2em}

We wish to replace loads and stores:
\begin{center}
\begin{scriptsize}
\begin{tt}
\begin{tabular}{l@{~~~}l@{~~~}l}
c = a + b;      &        iload\_1      &      ld [fp-32],R1\\
                &                     &      st R1,[fp-44]\\
                &        iload\_2      &      ld [fp-36],R1\\
                &                     &      st R1,[fp-48]\\
                &        iadd         &      ld [fp-48],R1\\
                &                     &      ld [fp-44],R2\\
                &                     &      add R2,R1,R1\\
                &                     &      st R1,[fp-44]\\
                &        istore\_3     &      ld [fp-44],R1\\
                &                     &      st R1,[fp-40]
\end{tabular}
\end{tt}
\end{scriptsize}
\end{center}

by registers operations:

\begin{center}
\begin{scriptsize}
\begin{tt}
\begin{tabular}{l@{~~~}l@{~~~}l}
c = a + b;       &       iload\_1     &       ld [fp-32],R1\\
                 &       iload\_2     &       ld [fp-36],R2\\
                 &       iadd        &       add R1,R2,R1\\
                 &       istore\_3    &       st R1,[fp-40]
\end{tabular}
\end{tt}
\end{scriptsize}
\end{center}
where {\tt R1} and {\tt R2} represent the stack.
\vfil
\renewcommand{\arraystretch}{1}
\end{slide*}
 
\begin{slide*}
\renewcommand{\arraystretch}{0.9}
The {\em fixed\/} register allocation scheme:
\begin{itemize}
\item assign $m$ registers to the first $m$ locals;
\item assign $n$ registers to the first $n$ stack locations;
\item assign $k$ scratch registers; and
\item spill remaining locals and locations into memory.
\end{itemize}
\vspace*{2em}
Example for 6 registers ($m=n=k=2$):
\begin{center}
\begin{scriptsize}
\begin{tt}
\begin{tabular}{l@{~~~}l@{~~~}l@{~~~}l}
name & offset & location & register \\
\\
              a    & 1  &           &  R1 \\
              b    & 2  &           &  R2 \\
              c    & 3  &  [fp-40]  &   \\
  \\
            stack  & 0  &           &  R3 \\
            stack  & 1  &           &  R4 \\
  \\
            scratch& 0  &           &  R5 \\
            scratch& 1  &           &  R6
\end{tabular}
\end{tt}
\end{scriptsize}
\end{center}
\vfil
\renewcommand{\arraystretch}{1}
\end{slide*}
 
\begin{slide*}
\renewcommand{\arraystretch}{0.9}
Improved native code generation:
\vspace*{0.5ex}
 
\begin{center}
\begin{scriptsize}
\begin{tt}
\begin{tabular}{l@{~~~}l@{~~~}l}
                 &                    &      save sp,-136,sp\\
a = 1;           &       iconst\_1     &      mov 1,R3\\
                 &       istore\_1     &      mov R3,R1\\
b = 13;          &       ldc 13    &      mov 13,R3\\
                 &       istore\_2     &      mov R3,R2\\
c = a + b;       &       iload\_1      &      mov R1,R3\\
                 &       iload\_2      &      mov R2,R4\\
                 &       iadd         &      add R3,R4,R3\\
                 &       istore\_3     &      st R3,[fp-40]\\
                 &       return       &      restore\\
                 &                    &      ret
\end{tabular}
\end{tt}
\end{scriptsize}
\end{center}
\vspace*{2em}

This works quite well if:
\begin{itemize}
\item the architecture has a large register set;
\item the stack is small most of the time; and
\item the first locals are used most frequently.
\end{itemize}
\renewcommand{\arraystretch}{1}
\vfil
\end{slide*}
 
\begin{slide*}
Summary of fixed register allocation scheme:

\begin{itemize}
\item registers are allocated once; and
\item the allocation does not change within a method.
\end{itemize}

Advantages:
\begin{itemize}
\item it's simple to do the allocation; and
\item no problems with different control flow paths.
\end{itemize}

Disadvantages:
\begin{itemize}
\item assumes the first locals and stack locations are most important; and
\item may waste registers within a region of a method.
\end{itemize}

\vfil
\end{slide*}
 
\begin{slide*}
The {\em basic block\/} register allocation scheme:

\begin{itemize}
\item assign frame slots to registers on demand within a basic block; and
\item update {\em descriptors\/} at each bytecode.
\end{itemize}
\vspace*{2em}

The descriptor maps a slot to an element of the set 
\{$\bot$, {\tt mem}, {\tt R$i$}, {\tt mem\&R$i$}\}:
\begin{scriptsize}
\begin{tt}
\begin{center}
\begin{tabular}{|l|c|}
\hline
a & R2 \\\hline
b & mem \\\hline
c & mem\&R4 \\\hline
s\_0 & R1 \\\hline
s\_1 & $\bot$ \\\hline
\end{tabular}
\end{center}
\end{tt}
\end{scriptsize}

We also maintain the inverse register map:
\begin{scriptsize}
\begin{tt}
\begin{center}
\begin{tabular}{|l|c|}
\hline
R1 & s\_0 \\\hline
R2 & a\\\hline
R3 & $\bot$\\\hline
R4 & c\\\hline
R5 & $\bot$\\\hline
\end{tabular}
\end{center}
\end{tt}
\end{scriptsize}
\vfil
\end{slide*}
 
\begin{slide*}
At the beginning of a basic block, all slots are in memory.\\

Basic blocks are merged by control paths:
\begin{center}
\hspace*{-3em}\setlength{\unitlength}{2000sp}%
%
\begingroup\makeatletter\ifx\SetFigFont\undefined%
\gdef\SetFigFont#1#2#3#4#5{%
  \reset@font\fontsize{#1}{#2pt}%
  \fontfamily{#3}\fontseries{#4}\fontshape{#5}%
  \selectfont}%
\fi\endgroup%
\begin{picture}(1237,2000)(2776,-3586)
\thinlines
\put(3001,-2461){\line( 2,-3){600}}
\put(3901,-3361){\line( 2, 3){600}}
\put(2000,-2130){\makebox(0,0)[lb]{\smash{\SetFigFont{12}{14.4}{\ttdefault}{\mddefault}{\updefault}\begin{scriptsize}\begin{tabular}{|l|c|}
\hline
a & R1\\\hline
b & R2 \\\hline
\end{tabular}\end{scriptsize}
}}}
\put(4351,-2130){\makebox(0,0)[lb]{\smash{\SetFigFont{8}{14.4}{\ttdefault}{\mddefault}{\updefault}\begin{scriptsize}\begin{tabular}{|l|c|}
\hline
a & R3\\\hline
b & R4 \\\hline
\end{tabular}\end{scriptsize}}}}
\put(3220,-3820){\makebox(0,0)[lb]{\smash{\SetFigFont{8}{14.4}{\ttdefault}{\mddefault}{\updefault}\begin{scriptsize}\begin{tabular}{|l|c|}
\hline
a & ?\\\hline
b & ? \\\hline
\end{tabular}\end{scriptsize}}}}
\end{picture}\\
~\\
\end{center}
\vspace*{2em}

Registers must be spilled after basic blocks:
\begin{center}
~\\
\hspace*{-3em}\setlength{\unitlength}{2000sp}%
%
\begingroup\makeatletter\ifx\SetFigFont\undefined%
\gdef\SetFigFont#1#2#3#4#5{%
  \reset@font\fontsize{#1}{#2pt}%
  \fontfamily{#3}\fontseries{#4}\fontshape{#5}%
  \selectfont}%
\fi\endgroup%
\begin{picture}(1237,2000)(2776,-3586)
\thinlines
\put(3001,-2461){\line( 2,-3){600}}
\put(3901,-3361){\line( 2, 3){600}}
\put(2000,-1900){\makebox(0,0)[lb]{\smash{\SetFigFont{12}{14.4}{\ttdefault}{\mddefault}{\updefault
}
\begin{minipage}{20em}
\begin{scriptsize}\begin{tabular}{|l|c|}
\hline
a & R1\\\hline
b & R2 \\\hline
\end{tabular}\\
\hspace*{-4ex}st R1,[fp-32]\\[-3ex]
\hspace*{-4ex}st R2,[fp-36]
\end{scriptsize}
\end{minipage}
}}}
\put(4351,-1870){\makebox(0,0)[lb]{\smash{\SetFigFont{8}{14.4}{\ttdefault}{\mddefault}{\updefault}\begin{minipage}{20em}
\begin{scriptsize}\begin{tabular}{|l|c|}
\hline
a & R3\\\hline
b & R4 \\\hline
\end{tabular}\\
st R3,[fp-32]\\[-3ex]
st R4,[fp-36]
\end{scriptsize}
\end{minipage}
}}}
\put(2940,-3820){\makebox(0,0)[lb]{\smash{\SetFigFont{8}{14.4}{\ttdefault}{\mddefault}{\updefault}
\begin{scriptsize}\begin{tabular}{|l|c|}
\hline
a & mem\\\hline
b & mem \\\hline
\end{tabular}\end{scriptsize}}}}
\end{picture}
\end{center}
\vfil
\end{slide*}

\begin{slide*}
\begin{scriptsize}
\begin{tt}
\begin{tabular}{l@{~~~~~}l@{~~~}l@{~~~}l}
 & save sp,-136,sp & {\tiny \begin{tabular}{|l|c|} 
                     \hline
                     R1 & $\bot$ \\\hline
                     R2 & $\bot$ \\\hline
                     R3 & $\bot$ \\\hline
                     R4 & $\bot$ \\\hline
                     R5 & $\bot$ \\\hline
                     \end{tabular}} &
                     {\tiny \begin{tabular}{|l|c|}
                     \hline
                     a & mem \\\hline
                     b & mem \\\hline
                     c & mem \\\hline
                     s\_0 & $\bot$ \\\hline
                     s\_1 & $\bot$ \\\hline
                     \end{tabular}}\\\\
iconst\_1  & mov 1,R1 & {\tiny \begin{tabular}{|l|c|} 
                     \hline
                     R1 & s\_0 \\\hline
                     R2 & $\bot$ \\\hline
                     R3 & $\bot$ \\\hline
                     R4 & $\bot$ \\\hline
                     R5 & $\bot$ \\\hline
                     \end{tabular}} &
                     {\tiny \begin{tabular}{|l|c|}
                     \hline
                     a & mem \\\hline
                     b & mem \\\hline
                     c & mem \\\hline
                     s\_0 & R1 \\\hline
                     s\_1 & $\bot$ \\\hline
                     \end{tabular}}\\\\
istore\_1  & mov R1,R2 & {\tiny \begin{tabular}{|l|c|} 
                     \hline
                     R1 & $\bot$ \\\hline
                     R2 & a \\\hline
                     R3 & $\bot$ \\\hline
                     R4 & $\bot$ \\\hline
                     R5 & $\bot$ \\\hline
                     \end{tabular}} &
                     {\tiny \begin{tabular}{|l|c|}
                     \hline
                     a & R2 \\\hline
                     b & mem \\\hline
                     c & mem \\\hline
                     s\_0 & $\bot$ \\\hline
                     s\_1 & $\bot$ \\\hline
                     \end{tabular}}\\\\
ldc 13  & mov 13,R1 & {\tiny \begin{tabular}{|l|c|} 
                     \hline
                     R1 & s\_0 \\\hline
                     R2 & a \\\hline
                     R3 & $\bot$ \\\hline
                     R4 & $\bot$ \\\hline
                     R5 & $\bot$ \\\hline
                     \end{tabular}} &
                     {\tiny \begin{tabular}{|l|c|}
                     \hline
                     a & R2 \\\hline
                     b & mem \\\hline
                     c & mem \\\hline
                     s\_0 & R1 \\\hline
                     s\_1 & $\bot$ \\\hline
                     \end{tabular}}\\\\
istore\_2  & mov R1,R3 & {\tiny \begin{tabular}{|l|c|} 
                     \hline
                     R1 & $\bot$ \\\hline
                     R2 & a \\\hline
                     R3 & b \\\hline
                     R4 & $\bot$ \\\hline
                     R5 & $\bot$ \\\hline
                     \end{tabular}} &
                     {\tiny \begin{tabular}{|l|l|}
                     \hline
                     a & R2 \\\hline
                     b & R3 \\\hline
                     c & mem \\\hline
                     s\_0 & $\bot$ \\\hline
                     s\_1 & $\bot$ \\\hline
                     \end{tabular}}
\end{tabular}
\end{tt}
\end{scriptsize}
\vfil
\end{slide*}

\begin{slide*}
\begin{scriptsize}
\begin{tt}
\begin{tabular}{l@{~~~~~}l@{~~~}l@{~~~}l}
iload\_1  & mov R2,R1 & {\tiny \begin{tabular}{|l|c|}
                     \hline
                     R1 & s\_0 \\\hline
                     R2 & a \\\hline
                     R3 & b \\\hline
                     R4 & $\bot$ \\\hline
                     R5 & $\bot$ \\\hline
                     \end{tabular}} &
                     {\tiny \begin{tabular}{|l|c|}
                     \hline
                     a & R2 \\\hline
                     b & R3 \\\hline
                     c & mem \\\hline
                     s\_0 & R1 \\\hline
                     s\_1 & $\bot$ \\\hline
                     \end{tabular}}\\\\
iload\_2  & mov R3,R4 & {\tiny \begin{tabular}{|l|c|}
                     \hline
                     R1 & s\_0 \\\hline
                     R2 & a \\\hline
                     R3 & b \\\hline
                     R4 & s\_1 \\\hline
                     R5 & $\bot$ \\\hline
                     \end{tabular}} &
                     {\tiny \begin{tabular}{|l|c|}
                     \hline
                     a & R2 \\\hline
                     b & R3 \\\hline
                     c & mem \\\hline
                     s\_0 & R1 \\\hline
                     s\_1 & R4 \\\hline
                     \end{tabular}}\\\\
iadd  & add R1,R4,R1 & {\tiny \begin{tabular}{|l|c|}
                     \hline
                     R1 & s\_0 \\\hline
                     R2 & a \\\hline
                     R3 & b \\\hline
                     R4 & $\bot$ \\\hline
                     R5 & $\bot$ \\\hline
                     \end{tabular}} &
                     {\tiny \begin{tabular}{|l|c|}
                     \hline
                     a & R2 \\\hline
                     b & R3 \\\hline
                     c & mem \\\hline
                     s\_0 & R1 \\\hline
                     s\_1 & $\bot$ \\\hline
                     \end{tabular}}\\\\
istore\_3  & st R1,R4 & {\tiny \begin{tabular}{|l|c|}
                     \hline
                     R1 & $\bot$ \\\hline
                     R2 & a \\\hline
                     R3 & b \\\hline
                     R4 & c \\\hline
                     R5 & $\bot$ \\\hline
                     \end{tabular}} &
                     {\tiny \begin{tabular}{|l|c|}
                     \hline
                     a & R2 \\\hline
                     b & R3 \\\hline
                     c & R4 \\\hline
                     s\_0 & $\bot$ \\\hline
                     s\_1 & $\bot$ \\\hline
                     \end{tabular}}\\\\
&\hspace*{-2ex}\begin{tabular}{l}
st R2,[fp-32]\\
st R3,[fp-36]\\
st R4,[fp-40]
\end{tabular} & {\tiny \begin{tabular}{|l|c|}
                     \hline
                     R1 & $\bot$ \\\hline
                     R2 & $\bot$ \\\hline
                     R3 & $\bot$ \\\hline
                     R4 & $\bot$ \\\hline
                     R5 & $\bot$ \\\hline
                     \end{tabular}} &
                     {\tiny \begin{tabular}{|l|c|}
                     \hline
                     a & mem \\\hline
                     b & mem \\\hline
                     c & mem \\\hline
                     s\_0 & $\bot$ \\\hline
                     s\_1 & $\bot$ \\\hline
                     \end{tabular}}\vspace{3ex}\\
 return & restore & & \\
        & ret & &
\end{tabular}
\end{tt}
\end{scriptsize}
\vfil
\end{slide*}

\begin{slide*}
So far, this is actually no better than the fixed scheme.\\

But if we add the statement:

\begin{scriptsize}
\begin{verbatim}
c = c * c + c;
\end{verbatim}
\end{scriptsize}

then the fixed scheme and basic block scheme generate:

\begin{scriptsize}
\begin{verbatim}
              Fixed            Basic block

iload_3    ld [fp-40],R3       mv R4, R1   
dup        ld [fp-40],R4       mv R4, R5
imul       mul R3,R4,R3        mul R1, R5, R1
iload_3    ld [fp-40],R4       mv R4, R5
iadd       add R3,R4,R3        add R1, R5, R1
istore_3   st R3,[fp-40]       mv R1, R4
\end{verbatim}
\end{scriptsize}

\vfil
\end{slide*}

\begin{slide*}
Summary of basic block register allocation scheme:

\begin{itemize}
\item registers are allocated on demand; and
\item slots are kept in registers within a basic block.
\end{itemize}
 
Advantages:
\begin{itemize}
\item registers are not wasted on unused slots; and
\item less spill code within a basic block.
\end{itemize}
 
Disadvantages:
\begin{itemize}
\item much more complex than the fixed register allocation scheme;
\item registers must be spilled at the end of a basic block; and
\item we may spill locals that are never needed.
\end{itemize}
\vfil
\end{slide*}
 
\begin{slide*}
\renewcommand{\arraystretch}{0.9}
We can optimize further:\\

\begin{center}
\begin{scriptsize}
\begin{tt}
\begin{tabular}{l@{~~~~~~~}l}
     save sp,-136,sp    &    save sp,-136,sp\\
 \\
     mov 1,R1           &    mov 1,R2\\
     mov R1,R2\\
 \\
     mov 13,R1          &    mov 13,R3\\
     mov R1,R3\\
 \\
     mov R2,R1            \\
     mov R3,R4\\
     add R1,R4,R1       &    add R2,R3,R1\\
     st R1,[fp-40]      &    st R1,[fp-40]\\
 \\
     restore            &    restore\\
     ret                &    ret
\end{tabular}
\end{tt}
\end{scriptsize}
\end{center}
\vspace*{2em}

by not explicitly modelling the stack.
\renewcommand{\arraystretch}{1}
\vfil
\end{slide*}
 
\begin{slide*}
Unfortunately, this cannot be done safely on the fly by a peephole optimizer.\\

The optimization:
\newcommand{\LONG}{\mbox{$\Longrightarrow$}}
\begin{tt}
\begin{tabbing}
XXXXXXXXXXXXX\=XXXXXX\=\kill
\\
mov 1,R3  \>   $\Longrightarrow$ \>   mov 1,R1 \\
mov R3,R1 \\

\end{tabbing}
\end{tt}
is unsound if {\tt R3} is used in a later instruction:
\begin{tt}
\begin{tabbing}
XXXXXXXXXXXXX\=XXXXXX\=\kill
\\
mov 1,R3  \> $\Longrightarrow$ \>  mov 1,R1 \\
mov R3,R1 \\
$\vdots$   \>\>               $\vdots$\\
mov R3,R4   \>\>        mov R3,R4\\
 
\end{tabbing}
\end{tt}
Such optimizations require dataflow analysis.
\vfil
\end{slide*}
 
\begin{slide*}
Invoking methods in bytecode:
\begin{itemize}
\item evaluate each argument leaving results on the stack; and
\item emit {\tt invokevirtual} instruction.
\end{itemize}
\vspace*{2em}

Invoking methods in native code:
\begin{itemize}
\item call library routine {\tt soft\_get\_method\_code} to perform the method lookup;
\item generate code to load arguments into registers; and
\item branch to the resolved address.
\end{itemize}
\vfil
\end{slide*}

\begin{slide*}
Consider a method invocation:\\

\begin{verbatim}
c = t.foo(a,b);

\end{verbatim}

where the memory map is:\\
\begin{center}
\begin{scriptsize}
\begin{tt}
\begin{tabular}{l@{~~~}l@{~~~}l@{~~~}l}
name & offset & location & register\\ \\
a & 1 & [fp-60] & R3 \\
b & 2 & [fp-56] & R4 \\
c & 3 & [fp-52] & \\
t & 4 & [fp-48] & R2\\
stack & 0 & [fp-36] & R1 \\
stack & 1 & [fp-40] & R5 \\
stack & 2 & [fp-44] & R6 \\
scratch & 0 & [fp-32] & R7\\
scratch & 1 & [fp-28] & R8
\end{tabular}
\end{tt}
\end{scriptsize}
\end{center}
\vfil
\end{slide*}

\begin{slide*}
\renewcommand{\arraystretch}{0.9}
Generating native code:
\begin{center}
\begin{scriptsize}
\begin{tt}
\begin{tabular}{l@{~~~~~~}l}
 aload\_4         &        mov R2,R1\\
 iload\_1         &        mov R3,R5\\
 iload\_2         &        mov R4,R6\\
 invokevirtual foo & \verb"//" soft call to get address \\
               &          ld R7,[R2+4]\\
               &          ld R8,[R7+52]\\
               &          \verb"//" spill all registers \\
               &          st R3,[fp-60]\\
               &          st R4,[fp-56]\\
               &          st R2,[fp-48]\\
               &          st R6,[fp-44]\\
               &          st R5,[fp-40]\\
               &          st R1,[fp-36]\\
               &          st R7,[fp-32]\\
               &          st R8,[fp-28]\\
               &          \verb"//" make call \\
               &          mov R8,R0\\
               &          call soft\_get\_method\_code\\
               &          \verb"//" result is in R0 \\
               &          \verb"//" put args in R2, R1, and R0\\
               &          ld R2,[fp-44]  \verb"//" R2 := stack\_2\\
               &          ld R1,[fp-40]  \verb"//" R1 := stack\_1 \\
               &          st R0,[fp-32]  \verb"//" spill result \\
               &          ld R0,[fp-36]  \verb"//" R0 := stack\_0 \\
               &          ld R4,[fp-32]  \verb"//" reload result \\
               &          jmp [R4]  \verb"//" call method 
\end{tabular}
\end{tt}
\end{scriptsize}
\end{center}
\begin{itemize}
\item this is long and costly; and
\item the lack of dataflow analysis causes massive spills within basic blocks.
\end{itemize}
\renewcommand{\arraystretch}{1}
\vfil
\end{slide*}

\begin{slide*}
Handling branches:
\begin{itemize}
\item the only problem is that the target
  address is not known;
\item assemblers normally handle this; but
\item the JIT compiler produces binary code directly in memory.
\end{itemize}
\renewcommand{\arraystretch}{0.9}

Generating native code:
\begin{center}
\begin{scriptsize}
\begin{tt}
\begin{tabular}{l@{~~~}l@{~~~}l}
if (a < b)     &        iload\_1        &   ld R1,[fp-44] \\
               &        iload\_2        &   ld R2,[fp-48] \\
               &        if\_icmpge 17   &   sub R1,R2,R3        \\
               &                       &   bge ??
\end{tabular}
\end{tt}
\end{scriptsize}
\end{center}
How to compute the branch targets:
\begin{itemize}
\item previously encountered branch targets are already known;
\item keep unresolved branches in a table; and
\item patch targets when the bytecode is eventually reached.
\end{itemize}
\renewcommand{\arraystretch}{1}
\vfil
\end{slide*}

\end{document}

