\documentclass[portrait]{seminar}
%\usepackage{pandora}
\usepackage{color}
\usepackage{fancybox}
\usepackage{alltt}
\usepackage{epsfig}
\usepackage{rail}
\usepackage{bar}
\usepackage{url}
\usepackage{rotating}
\usepackage[normalem]{ulem}
\usepackage{latexsym}
\usepackage{amsmath}

\begin{document}

\boldmath
\newcommand{\RA}{$\rightarrow$}
\newcommand{\LL}{\mbox{$[$\hspace{-0.15em}$[$}}
\newcommand{\RR}{\mbox{$]$\hspace{-0.15em}$]$}}
\newcommand{\CC}[1]{\mbox{\tt $\LL$#1$\RR$}}

\slideframe{shadow}

%%% Activate one of these to get either Aarhus style or McGill style 
%%% by putting a #1 in the appropriate line.
\newcommand{\mcgill}[1]{#1}
\newcommand{\aarhus}[1]{}

%%% Define this to be the name of your term
\newcommand{\courseterm}{Fall 2012}




\aarhus{
\newpagestyle{dOvsstyle}{dOvs'98 Week 49 \hfil Static analysis}{\hfil \thepage}
}

\mcgill{
\newpagestyle{dOvsstyle}{COMP 520 \courseterm  \hfil Static analysis (\thepage)}{}
}
 
\slidepagestyle{dOvsstyle}

\begin{slide*}
\begin{tabbing}
\aarhus{{\Large\bf Week 49}\\}
~\\
{\Huge\bf Static analysis}\\
\end{tabbing}
\begin{center}
\setlength{\unitlength}{2300sp}%
%
\begingroup\makeatletter\ifx\SetFigFont\undefined%
\gdef\SetFigFont#1#2#3#4#5{%
  \reset@font\fontsize{#1}{#2pt}%
  \fontfamily{#3}\fontseries{#4}\fontshape{#5}%
  \selectfont}%
\fi\endgroup%
\begin{picture}(4824,5805)(1489,-5386)
\thinlines
\put(1576,-1036){\line( 1, 0){ 75}}
\put(1726,-1036){\line( 1, 0){225}}
\put(1576,-1111){\line( 1, 0){225}}
\put(1876,-1111){\line( 1, 0){150}}
\put(1576,-1186){\line( 1, 0){150}}
\put(1801,-1186){\line( 1, 0){225}}
\put(1576,-1261){\line( 1, 0){ 75}}
\put(1726,-1261){\line( 1, 0){ 75}}
\put(1876,-1261){\line( 1, 0){150}}
\put(1576,-1336){\line( 1, 0){ 75}}
\put(1726,-1336){\line( 1, 0){225}}
\put(1576,-1411){\line( 1, 0){225}}
\put(1876,-1411){\line( 1, 0){150}}
\put(1576,-1486){\line( 1, 0){150}}
\put(1801,-1486){\line( 1, 0){225}}
\put(1576,-1561){\line( 1, 0){ 75}}
\put(1726,-1561){\line( 1, 0){ 75}}
\put(1876,-1561){\line( 1, 0){150}}
\put(1576,-1636){\line( 1, 0){ 75}}
\put(1726,-1636){\line( 1, 0){225}}
\put(1576,-1711){\line( 1, 0){225}}
\put(1876,-1711){\line( 1, 0){150}}
\put(1576,-1786){\line( 1, 0){150}}
\put(1801,-1786){\line( 1, 0){225}}
\put(1576,-1861){\line( 1, 0){ 75}}
\put(1726,-1861){\line( 1, 0){ 75}}
\put(1876,-1861){\line( 1, 0){150}}
\put(1576,-2836){\line( 1, 0){ 75}}
\put(1726,-2836){\line( 1, 0){225}}
\put(1576,-2911){\line( 1, 0){225}}
\put(1876,-2911){\line( 1, 0){150}}
\put(1576,-2986){\line( 1, 0){150}}
\put(1801,-2986){\line( 1, 0){225}}
\put(1576,-3061){\line( 1, 0){ 75}}
\put(1726,-3061){\line( 1, 0){ 75}}
\put(1876,-3061){\line( 1, 0){150}}
\put(1576,-3136){\line( 1, 0){ 75}}
\put(1726,-3136){\line( 1, 0){225}}
\put(1576,-3211){\line( 1, 0){225}}
\put(1876,-3211){\line( 1, 0){150}}
\put(1576,-3286){\line( 1, 0){150}}
\put(1801,-3286){\line( 1, 0){225}}
\put(1576,-3361){\line( 1, 0){ 75}}
\put(1726,-3361){\line( 1, 0){ 75}}
\put(1876,-3361){\line( 1, 0){150}}
\put(1576,-4561){\line( 1, 0){ 75}}
\put(1726,-4561){\line( 1, 0){225}}
\put(1576,-4636){\line( 1, 0){225}}
\put(1876,-4636){\line( 1, 0){150}}
\put(1576,-4711){\line( 1, 0){150}}
\put(1801,-4711){\line( 1, 0){225}}
\put(1576,-4786){\line( 1, 0){ 75}}
\put(1726,-4786){\line( 1, 0){ 75}}
\put(1876,-4786){\line( 1, 0){150}}
\put(1801,-2761){\vector( 0, 1){600}}
\put(1576,-1936){\line( 1, 0){300}}
\put(1576,-2011){\line( 1, 0){ 75}}
\put(1726,-2011){\line( 1, 0){300}}
\put(1576,-2086){\line( 1, 0){150}}
\put(1801,-2086){\line( 1, 0){ 75}}
\put(1951,-2086){\line( 1, 0){ 75}}
\put(1576,-3361){\line( 1, 0){ 75}}
\put(1726,-3361){\line( 1, 0){225}}
\put(1576,-3436){\line( 1, 0){225}}
\put(1876,-3436){\line( 1, 0){150}}
\put(1576,-3511){\line( 1, 0){150}}
\put(1801,-3511){\line( 1, 0){225}}
\put(1576,-3586){\line( 1, 0){ 75}}
\put(1726,-3586){\line( 1, 0){ 75}}
\put(1876,-3586){\line( 1, 0){150}}
\put(1576,-4336){\line( 1, 0){ 75}}
\put(1726,-4336){\line( 1, 0){225}}
\put(1576,-4411){\line( 1, 0){225}}
\put(1876,-4411){\line( 1, 0){150}}
\put(1576,-4486){\line( 1, 0){150}}
\put(1801,-4486){\line( 1, 0){225}}
\put(1576,-4561){\line( 1, 0){ 75}}
\put(1726,-4561){\line( 1, 0){ 75}}
\put(1876,-4561){\line( 1, 0){150}}
\put(4351,-3061){\circle*{120}}
\put(3976,-811){\circle*{120}}
\put(4576,-1411){\circle*{120}}
\put(4351,-2311){\circle*{120}}
\put(1501,-4861){\framebox(600,600){}}
\put(1501,-3661){\framebox(600,900){}}
\put(1501,-2161){\framebox(600,1200){}}
\put(1801,-4261){\vector( 0, 1){600}}
\put(2101,-4561){\vector( 3, 2){2250}}
\put(2101,-1561){\vector( 3,-2){2250}}
\put(2101,-3061){\vector( 1, 0){2175}}
\put(2101,-1561){\vector( 3,-1){2250}}
\put(4351,-3061){\line( 0, 1){750}}
\put(2101,-3061){\vector( 3, 2){2475}}
\put(4351,-2311){\line( 1, 4){225}}
\put(4576,-1411){\line(-1, 1){600}}
\put(2101,-4561){\vector( 1, 2){1875}}
\put(4501,239){\line(-2,-3){1800}}
\put(2701,-2461){\line( 2,-3){1800}}
\put(4501,-5161){\line( 2, 3){1800}}
\put(6301,-2461){\line(-2, 3){1800}}
\put(4400,314){\makebox(0,0)[lb]{\smash{\SetFigFont{8}{14.4}{\familydefault}{\mddefault}{\updefault}$\top$}}}
\put(4390,-5410){\makebox(0,0)[lb]{\smash{\SetFigFont{8}{14.4}{\familydefault}{\mddefault}{\updefault}$\bot$}}}
\end{picture}
\end{center}
\vfil
\end{slide*}

\begin{slide*}
{\em Static analysis\/} determines interesting properties of programs to enable
some optimizations.

All interesting properties are actually undecidable, so the analysis computes
a conservative approximation:
\begin{itemize}
\item if we say {\em yes}, then the property definitely holds;
\item if we say {\em no}, then the property may or may not hold;
\item only the {\em yes\/} answer will help us to perform the optimization;
\item a trivial analysis will say {\em no\/} always; so
\item the art is to say {\em yes\/} as often as possible.
\end{itemize}
Properties need not be simply {\em yes\/} or {\em no}, in which case the
notion of {\em approximation\/} is more subtle.
\vfil
\end{slide*}
 
\begin{slide*}
Static analysis may take place:
\begin{itemize}
\item at the source code level;
\item at some intermediate level; or
\item at the machine code level.
\end{itemize}
\vspace*{2em}

Static analysis may look at:
\begin{itemize}
\item basic blocks only;
\item an entire function (intraprocedural); or
\item the whole program (interprocedural).
\end{itemize}
In each case, we are maximally pessimistic at the boundaries.\\

The precision and cost of an analysis rises as we include more information.
\vfil
\end{slide*}

\begin{slide*}
Simple static analysis:
\begin{itemize}
\item is merely advanced weeding;
\item uses symbol and type information; and
\item is recursive in the program syntax.
\end{itemize}
\vspace*{2em}

An example is the {\em definite assignment\/} requirement in Java and JOOS:
\begin{itemize}
\item local variables must be assigned before they are read;
\item this is undecidable; but
\item the language specification dictates a specific conservative approximation.
\end{itemize}
\vfil
\end{slide*}

\begin{slide*}
For each program point, compute a set of local variables that:
\begin{itemize}
\item contains only variables that have definitely been assigned;
\item may be too small, since the analysis is conservative; and
\item depends on the set computed for the previous program point.
\end{itemize}
\vspace*{2em}

It accepts:
\begin{scriptsize}
\begin{verbatim}
{ int k;
  if (flag) k = 3; else k = 4;
  System.out.println(k);
}
\end{verbatim}
\end{scriptsize}
but rejects:
\begin{scriptsize}
\begin{verbatim}
{ int k;
  if (flag) k = 3; 
  if (!flag) k = 4;
  System.out.println(k);
}
\end{verbatim}
\end{scriptsize}
\vfil
\end{slide*}

\begin{slide*}
JOOS code for statements:
\begin{scriptsize}
\begin{verbatim}
ASNSET *defasnSTATEMENT(STATEMENT *s, ASNSET *before)
{ if (s!=NULL) {
     switch (s->kind) {
       case skipK:
            return before;
       case expK:
            return defasnEXP(s->val.expS,before);
       case returnK:
            if (s->val.returnS!=NULL)
               (void)defasnEXP(s->val.returnS,before);
            return setUniversal();
       case sequenceK:
            return 
              defasnSTATEMENT(s->val.sequenceS.second,
                 defasnSTATEMENT(s->val.sequenceS.first,
                                 before)
              );
       case ifelseK:
            return 
              setIntersect(
                defasnSTATEMENT(s->val.ifelseS.thenpart,
                  defasnEXPassume(s->val.ifelseS.condition,
                                  before,1)
                ),
                defasnSTATEMENT(s->val.ifelseS.elsepart,
                  defasnEXPassume(s->val.ifelseS.condition,
                                  before,0)
                )
              );
       case ...
  } }
\end{verbatim}
\end{scriptsize}
\vfil
\end{slide*}

\begin{slide*}
To make the analysis more precise, it considers boolean expressions in more detail.\\

The procedure {\tt defasnEXPassume(...,b)} assumes the expression evalutes to {\tt b}.\\

This refinement handles a case like:
\begin{scriptsize}
\begin{verbatim}
{ int k;
  if (a>0 && (k=b)>0) System.out.println(k);
}
\end{verbatim}
\end{scriptsize}
which would otherwise be rejected.\\

In general, a static analysis becomes more precise when it may
make further assumptions about the context. 
\vfil
\end{slide*}

\begin{slide*}
The definite assignment analysis is particularly simple:

there are no recursive dependencies between the computed sets.\\

This allows a simple implementation:\\ a top-down traversal of the
parse tree.\\

For more sophisticated analyses, we generate equations and compute the solution
as a fixed point.
\vfil
\end{slide*}

\begin{slide*}
For the JIT compiler, we want to optimize the use of registers:
\begin{tt}
\begin{tabbing}
XXXXXXXXXXXXX\=XXXXXX\=\kill
\\
mov 1,R3  \>   $\Longrightarrow$ \>   mov 1,R1 \\
mov R3,R1 \\
 
\end{tabbing}
\end{tt}
This requires knowledge about the future uses of registers:\\

The optimization is only sound if the value of {\tt R3} is not used
later on.

\vfil
\end{slide*}
 
\begin{slide*}
For basic block register allocation,  which variables
need to be written back to memory?

The na\"ive scheme:
\begin{itemize}
\item must write all those variables that are {\em only\/} in
registers.
\end{itemize}

A better scheme:
\begin{itemize}
\item write all those variables that are only in registers {\em and\/} whose
values might be used later on.
\end{itemize}

This could avoid many useless spills.
\vfil
\end{slide*}

\begin{slide*}
In both examples, we need to know if some {\tt R$_i$} might be used later on.
If so, it is called {\em live}; otherwise, it is called {\em dead}.\\

A static analysis can conservatively approximate liveness at each program point.

Exact liveness is of course undecidable.\\

A trivial analysis will call everything live, which precludes all optimizations.

A superior analysis will identify more dead variables.
\vfil
\end{slide*}
 
\begin{slide*}
Liveness analysis for VirtualRISC:

\begin{itemize}
\item build a control flow graph (goto graph);
\item define dataflow equations for each node;
\item compute the least solution of these equations.
\end{itemize}

For basic blocks the computation is trivial.

For intraprocedural analysis we must compute a minimal fixed point in a lattice.
\vfil
\end{slide*}
 
\begin{slide*}
Consider a simple basic block:

\begin{scriptsize}
\begin{alltt}

mov 3,\underline{R1}
mov 4,\underline{R2}
add R1,R2,\underline{R3}
mov R3,\underline{R0}
return

\end{alltt}
\end{scriptsize}
The \underline{underlined} registers are written (defined), the others are merely read
(used).

The control flow graph is:\\

~~~~~~~~~~~~~~~~~~\setlength{\unitlength}{2000sp}%
%
\begingroup\makeatletter\ifx\SetFigFont\undefined%
\gdef\SetFigFont#1#2#3#4#5{%
  \reset@font\fontsize{#1}{#2pt}%
  \fontfamily{#3}\fontseries{#4}\fontshape{#5}%
  \selectfont}%
\fi\endgroup%
\begin{picture}(912,3750)(2401,-5086)
\thinlines
\put(3301,-1561){\vector( 0,-1){600}}
\put(3301,-2461){\vector( 0,-1){600}}
\put(3301,-3361){\vector( 0,-1){600}}
\put(3301,-4261){\vector( 0,-1){600}}
\put(2401,-1486){\makebox(0,0)[lb]{\smash{\SetFigFont{8}{14.4}{\ttdefault}{\mddefault}{\updefault}S1:  mov 3,\underline{R1}}}}
\put(2401,-2386){\makebox(0,0)[lb]{\smash{\SetFigFont{8}{14.4}{\ttdefault}{\mddefault}{\updefault}S2:  mov 4,\underline{R2}}}}
\put(2401,-3286){\makebox(0,0)[lb]{\smash{\SetFigFont{8}{14.4}{\ttdefault}{\mddefault}{\updefault}S3:  add R1,R2,\underline{R3}}}}
\put(2401,-4186){\makebox(0,0)[lb]{\smash{\SetFigFont{8}{14.4}{\ttdefault}{\mddefault}{\updefault}S4:  mov R3,\underline{R0}}}}
\put(2401,-5086){\makebox(0,0)[lb]{\smash{\SetFigFont{8}{14.4}{\ttdefault}{\mddefault}{\updefault}S5:  return}}}
\end{picture}
\vfil
\end{slide*}
 
\begin{slide*}
Each instruction uses some registers and defines some registers:\\

\setlength{\unitlength}{2000sp}%
%
\begingroup\makeatletter\ifx\SetFigFont\undefined%
\gdef\SetFigFont#1#2#3#4#5{%
  \reset@font\fontsize{#1}{#2pt}%
  \fontfamily{#3}\fontseries{#4}\fontshape{#5}%
  \selectfont}%
\fi\endgroup%
\begin{picture}(4800,4290)(2401,-5086)
\put(5401,-961){\makebox(0,0)[lb]{\smash{\SetFigFont{8}{14.4}{\ttdefault}{\mddefault}{\updefault}{\rm uses({\tt S}$_i$)}}}}
\put(7201,-961){\makebox(0,0)[lb]{\smash{\SetFigFont{8}{14.4}{\ttdefault}{\mddefault}{\updefault}{\rm defines({\tt S}$_i$)}}}}
\put(5401,-1486){\makebox(0,0)[lb]{\smash{\SetFigFont{8}{14.4}{\ttdefault}{\mddefault}{\updefault}\{\}}}}
\put(7201,-1486){\makebox(0,0)[lb]{\smash{\SetFigFont{8}{14.4}{\ttdefault}{\mddefault}{\updefault}\{R1\}}}}
\put(5401,-2386){\makebox(0,0)[lb]{\smash{\SetFigFont{8}{14.4}{\ttdefault}{\mddefault}{\updefault}\{\}}}}
\put(7201,-2386){\makebox(0,0)[lb]{\smash{\SetFigFont{8}{14.4}{\ttdefault}{\mddefault}{\updefault}\{R2\}}}}
\put(5401,-3286){\makebox(0,0)[lb]{\smash{\SetFigFont{8}{14.4}{\ttdefault}{\mddefault}{\updefault}\{R1,R2\}}}}
\put(7201,-3286){\makebox(0,0)[lb]{\smash{\SetFigFont{8}{14.4}{\ttdefault}{\mddefault}{\updefault}\{R3\}}}}
\put(5401,-4186){\makebox(0,0)[lb]{\smash{\SetFigFont{8}{14.4}{\ttdefault}{\mddefault}{\updefault}\{R3\}}}}
\put(7201,-4186){\makebox(0,0)[lb]{\smash{\SetFigFont{8}{14.4}{\ttdefault}{\mddefault}{\updefault}\{R0\}}}}
\put(5401,-5086){\makebox(0,0)[lb]{\smash{\SetFigFont{8}{14.4}{\ttdefault}{\mddefault}{\updefault}\{R0\}}}}
\put(7201,-5086){\makebox(0,0)[lb]{\smash{\SetFigFont{8}{14.4}{\ttdefault}{\mddefault}{\updefault}\{\}}}}
\thinlines
\put(3301,-1561){\vector( 0,-1){600}}
\put(3301,-2461){\vector( 0,-1){600}}
\put(3301,-3361){\vector( 0,-1){600}}
\put(3301,-4261){\vector( 0,-1){600}}
\put(2401,-1486){\makebox(0,0)[lb]{\smash{\SetFigFont{8}{14.4}{\ttdefault}{\mddefault}{\updefault}S1:  mov 3,\underline{R1}}}}
\put(2401,-2386){\makebox(0,0)[lb]{\smash{\SetFigFont{8}{14.4}{\ttdefault}{\mddefault}{\updefault}S2:  mov 4,\underline{R2}}}}
\put(2401,-3286){\makebox(0,0)[lb]{\smash{\SetFigFont{8}{14.4}{\ttdefault}{\mddefault}{\updefault}S3:  add R1,R2,\underline{R3}}}}
\put(2401,-4186){\makebox(0,0)[lb]{\smash{\SetFigFont{8}{14.4}{\ttdefault}{\mddefault}{\updefault}S4:  mov R3,\underline{R0}}}}
\put(2401,-5086){\makebox(0,0)[lb]{\smash{\SetFigFont{8}{14.4}{\ttdefault}{\mddefault}{\updefault}S5:  return}}}
\end{picture}\\

The register {\tt R0} is implicitly used for the return value.
\vfil
\end{slide*}
 
\begin{slide*}
Let out({\tt S}$_i$) be the variables that are live just {\em after\/} {\tt S}$_i$ and
in({\tt S}$_i$) those that are live just {\em before\/} {\tt S}$_i$:

~~~~~~~~~~~~~~~~~~\setlength{\unitlength}{2000sp}%
%
\begingroup\makeatletter\ifx\SetFigFont\undefined%
\gdef\SetFigFont#1#2#3#4#5{%
  \reset@font\fontsize{#1}{#2pt}%
  \fontfamily{#3}\fontseries{#4}\fontshape{#5}%
  \selectfont}%
\fi\endgroup%
\begin{picture}(1200,1524)(2401,-3973)
\thinlines
\put(3301,-2461){\vector( 0,-1){600}}
\put(3301,-3361){\vector( 0,-1){600}}
\put(2351,-3280){\makebox(0,0)[lb]{\smash{\SetFigFont{8}{14.4}{\ttdefault}{\mddefault}{\updefault}S$_i$:  op X,Y,Z}}}
\put(3601,-3661){\makebox(0,0)[lb]{\smash{\SetFigFont{8}{14.4}{\ttdefault}{\mddefault}{\updefault}{\rm out({\tt S}$_i$)}}}}
\put(3601,-2836){\makebox(0,0)[lb]{\smash{\SetFigFont{8}{14.4}{\ttdefault}{\mddefault}{\updefault}{\rm in({\tt S}$_i$)}}}}
\end{picture}\\

Then we have the dataflow equation:\\

in({\tt S}$_i$) = uses({\tt S}$_i$) $\cup$ (out({\tt S}$_i$) $-$ defines({\tt S}$_i$))\\

We add those registers that are used in the current instruction and delete those that are
defined here.
\vfil
\end{slide*}
 
\begin{slide*}
Since out({\tt S5}) = {\tt \{\}}, it follows that:\\

in({\tt S5}) = uses({\tt S5}) = {\tt \{R0\}}\\

We can continue to unravel the equations:
\begin{scriptsize}
\begin{tabbing}
XXX\=out({\tt S5}) \=\kill
\\
\>out({\tt S4})\>= in({\tt S5}) = {\tt \{R0\}}\\
\>in({\tt S4})\>= uses({\tt S4}) $\cup$ (out({\tt S4}) $-$ defines({\tt S4}))\\
\>            \>= {\tt \{R3\}} $\cup$ ({\tt \{R0\}} $-$ {\tt \{R0\}})\\
\>            \>= {\tt \{R3\}}\\
\>out({\tt S3})\>= in({\tt S4}) = {\tt \{R3\}}\\
\>in({\tt S3})\>= uses({\tt S3}) $\cup$ (out({\tt S3}) $-$ defines({\tt S3}))\\
\>            \>= {\tt \{R1,R2\}} $\cup$ ({\tt \{R3\}} $-$ {\tt \{R3\}})\\
\>            \>= {\tt \{R1,R2\}}
\end{tabbing}
\end{scriptsize}
and so on:\vspace*{1ex}\\
\setlength{\unitlength}{1800sp}%
%
\begingroup\makeatletter\ifx\SetFigFont\undefined%
\gdef\SetFigFont#1#2#3#4#5{%
  \reset@font\fontsize{#1}{#2pt}%
  \fontfamily{#3}\fontseries{#4}\fontshape{#5}%
  \selectfont}%
\fi\endgroup%
\begin{picture}(6600,4290)(2401,-5086)
\put(5401,-961){\makebox(0,0)[lb]{\smash{\SetFigFont{8}{14.4}{\ttdefault}{\mddefault}{\updefault}{\rm uses({\tt S$_i$})}}}}
\put(7201,-961){\makebox(0,0)[lb]{\smash{\SetFigFont{8}{14.4}{\ttdefault}{\mddefault}{\updefault}{\rm defines({\tt S$_i$})}}}}
\put(5401,-1486){\makebox(0,0)[lb]{\smash{\SetFigFont{8}{14.4}{\ttdefault}{\mddefault}{\updefault}\{\}}}}
\put(7201,-1486){\makebox(0,0)[lb]{\smash{\SetFigFont{8}{14.4}{\ttdefault}{\mddefault}{\updefault}\{R1\}}}}
\put(5401,-2386){\makebox(0,0)[lb]{\smash{\SetFigFont{8}{14.4}{\ttdefault}{\mddefault}{\updefault}\{\}}}}
\put(7201,-2386){\makebox(0,0)[lb]{\smash{\SetFigFont{8}{14.4}{\ttdefault}{\mddefault}{\updefault}\{R2\}}}}
\put(5401,-3286){\makebox(0,0)[lb]{\smash{\SetFigFont{8}{14.4}{\ttdefault}{\mddefault}{\updefault}\{R1,R2\}}}}
\put(7201,-3286){\makebox(0,0)[lb]{\smash{\SetFigFont{8}{14.4}{\ttdefault}{\mddefault}{\updefault}\{R3\}}}}
\put(5401,-4186){\makebox(0,0)[lb]{\smash{\SetFigFont{8}{14.4}{\ttdefault}{\mddefault}{\updefault}\{R3\}}}}
\put(7201,-4186){\makebox(0,0)[lb]{\smash{\SetFigFont{8}{14.4}{\ttdefault}{\mddefault}{\updefault}\{R0\}}}}
\put(5401,-5086){\makebox(0,0)[lb]{\smash{\SetFigFont{8}{14.4}{\ttdefault}{\mddefault}{\updefault}\{R0\}}}}
\put(7201,-5086){\makebox(0,0)[lb]{\smash{\SetFigFont{8}{14.4}{\ttdefault}{\mddefault}{\updefault}\{\}}}}
\put(3401,-1561){\vector( 0,-1){600}}
\put(3401,-2461){\vector( 0,-1){600}}
\put(3401,-3361){\vector( 0,-1){600}}
\put(3401,-4261){\vector( 0,-1){600}}
\put(2401,-1486){\makebox(0,0)[lb]{\smash{\SetFigFont{8}{14.4}{\ttdefault}{\mddefault}{\updefault}S1:  mov 3,\underline{R1}}}}
\put(2401,-2386){\makebox(0,0)[lb]{\smash{\SetFigFont{8}{14.4}{\ttdefault}{\mddefault}{\updefault}S2:  mov 4,\underline{R2}}}}
\put(2401,-3286){\makebox(0,0)[lb]{\smash{\SetFigFont{8}{14.4}{\ttdefault}{\mddefault}{\updefault}S3:  add R1,R2,\underline{R3}}}}
\put(2401,-4186){\makebox(0,0)[lb]{\smash{\SetFigFont{8}{14.4}{\ttdefault}{\mddefault}{\updefault}S4:  mov R3,\underline{R0}}}}
\put(2401,-5086){\makebox(0,0)[lb]{\smash{\SetFigFont{8}{14.4}{\ttdefault}{\mddefault}{\updefault}S5:  return}}}
\put(9001,-961){\makebox(0,0)[lb]{\smash{\SetFigFont{8}{14.4}{\ttdefault}{\mddefault}{\updefault}{\rm in({\tt S$_i$})}}}}
\put(9001,-1486){\makebox(0,0)[lb]{\smash{\SetFigFont{8}{14.4}{\ttdefault}{\mddefault}{\updefault}\{\}}}}
\put(9001,-2386){\makebox(0,0)[lb]{\smash{\SetFigFont{8}{14.4}{\ttdefault}{\mddefault}{\updefault}\{R1\}}}}
\put(9001,-3286){\makebox(0,0)[lb]{\smash{\SetFigFont{8}{14.4}{\ttdefault}{\mddefault}{\updefault}\{R1,R2\}}}}
\put(9001,-4186){\makebox(0,0)[lb]{\smash{\SetFigFont{8}{14.4}{\ttdefault}{\mddefault}{\updefault}\{R3\}}}}
\put(9001,-5086){\makebox(0,0)[lb]{\smash{\SetFigFont{8}{14.4}{\ttdefault}{\mddefault}{\updefault}\{R0\}}}}
\end{picture}
\vfil
\end{slide*}
 
\begin{slide*}
In basic blocks we use the equation:
$$\mbox{out({\tt S$_i$})} = \mbox{in({\tt S$_{i+1}$})}$$

If we have branches, then a node in the control flow graph may have
several successors.

In this case, we must use the equation:
$$\mbox{out({\tt S$_i$})} = \bigcup_{\displaystyle x \in \mbox{succ({\tt S$_i$})}}\mbox{in($x$)}$$

But now the equations are cyclic and cannot simply be unraveled.
\vfil
\end{slide*}
 
\begin{slide*}
Consider the small piece of C code:

\begin{scriptsize}
\begin{verbatim}
{ int i, sum_even, sum_odd, sum;
  i = 1;
  sum_even = 0;
  sum_odd = 0;
  sum = 0;
  while (i < 10)
   { if (i%2 == 0) sum_even = sum_even + i;
     else sum_odd = sum_odd + i;
     sum = sum + i;
     i++;
   }
}
\end{verbatim}
\end{scriptsize}
It yields the following VirtualRISC code:
\begin{scriptsize}
\begin{alltt}
        mov 1,\underline{R1}          // R1 is i
        mov 0,\underline{R2}          // R2 is sum_even
        mov 0,\underline{R3}          // R3 is sum_odd
        mov 0,\underline{R4}          // R4 is sum
loop:
        andcc R1,1,\underline{R5}    // R5 = R1 & 1
        cmp R5,0
        bne else         // if R5 != 0 goto else
        add R2,R1,\underline{R2}     // R2 = R2 + R1; even case
        b endif
else:
        add R3,R1,\underline{R3}     // R3 = R3 + R1; odd case
endif:
        add R4,R1,\underline{R4}     // R4 = R4 + R1; update sum
        add R1,1,\underline{R1}      // R1 = R1 + 1;  increment i
        cmp R1,9 
        ble loop         // if i <= 9 goto loop
\end{alltt}
\end{scriptsize}
\vfil
\end{slide*}
 
\begin{slide*}
The control flow graph:\\


~~~~~~~~~\setlength{\unitlength}{1700sp}%
%
\begingroup\makeatletter\ifx\SetFigFont\undefined%
\gdef\SetFigFont#1#2#3#4#5{%
  \reset@font\fontsize{#1}{#2pt}%
  \fontfamily{#3}\fontseries{#4}\fontshape{#5}%
  \selectfont}%
\fi\endgroup%
\begin{picture}(5712,10050)(901,-10486)
\thinlines
\put(3401,-661){\vector( 0,-1){600}}
\put(3401,-1561){\vector( 0,-1){600}}
\put(3401,-2461){\vector( 0,-1){600}}
\put(3401,-3361){\vector( 0,-1){600}}
\put(2401,-586){\makebox(0,0)[lb]{\smash{\SetFigFont{8}{14.4}{\ttdefault}{\mddefault}{\updefault}S1: mov 1,R1}}}
\put(2401,-1486){\makebox(0,0)[lb]{\smash{\SetFigFont{8}{14.4}{\ttdefault}{\mddefault}{\updefault}S2: mov 0,R2}}}
\put(2401,-2386){\makebox(0,0)[lb]{\smash{\SetFigFont{8}{14.4}{\ttdefault}{\mddefault}{\updefault}S3: mov 0,R3}}}
\put(2401,-3286){\makebox(0,0)[lb]{\smash{\SetFigFont{8}{14.4}{\ttdefault}{\mddefault}{\updefault}S4: mov 0,R4}}}
\put(2401,-4186){\makebox(0,0)[lb]{\smash{\SetFigFont{8}{14.4}{\ttdefault}{\mddefault}{\updefault}S5: andcc R1,1,R5}}}
\put(3401,-5161){\vector( 0,-1){600}}
\put(3401,-7861){\vector( 0,-1){600}}
\put(3401,-8761){\vector( 0,-1){600}}
\put(3401,-9661){\vector( 0,-1){600}}
\put(3001,-6061){\vector(-3,-2){900}}
\put(3601,-6061){\vector( 3,-2){900}}
\put(2101,-6961){\vector( 3,-2){900}}
\put(4501,-6961){\vector(-3,-2){900}}
\put(3401,-4261){\vector( 0,-1){600}}
\put(4301,-10411){\line( 1, 0){2300}}
\put(6601,-10411){\line( 0, 1){6300}}
\put(6601,-4111){\vector(-1, 0){1350}}
\put(2401,-5086){\makebox(0,0)[lb]{\smash{\SetFigFont{8}{14.4}{\ttdefault}{\mddefault}{\updefault}S6: cmp R5,0}}}
\put(2401,-5986){\makebox(0,0)[lb]{\smash{\SetFigFont{8}{14.4}{\ttdefault}{\mddefault}{\updefault}S7: bne S9}}}
\put(901,-6886){\makebox(0,0)[lb]{\smash{\SetFigFont{8}{14.4}{\ttdefault}{\mddefault}{\updefault}S8:~add R2,R1,R2}}}
\put(3901,-6886){\makebox(0,0)[lb]{\smash{\SetFigFont{8}{14.4}{\ttdefault}{\mddefault}{\updefault}S9:~add R3,R1,R3}}}
\put(2401,-7786){\makebox(0,0)[lb]{\smash{\SetFigFont{8}{14.4}{\ttdefault}{\mddefault}{\updefault}S10:~add R4,R1,R4}}}
\put(2401,-8686){\makebox(0,0)[lb]{\smash{\SetFigFont{8}{14.4}{\ttdefault}{\mddefault}{\updefault}S11:~add R1,1,R1}}}
\put(2401,-9586){\makebox(0,0)[lb]{\smash{\SetFigFont{8}{14.4}{\ttdefault}{\mddefault}{\updefault}S12:~cmp R1,9}}}
\put(2401,-10486){\makebox(0,0)[lb]{\smash{\SetFigFont{8}{14.4}{\ttdefault}{\mddefault}{\updefault}S13:~ble S5}}}
\end{picture}
\vfil
\end{slide*}

\begin{slide*}
To unravel the liveness equations, we should start with:
$$\mbox{out({\tt S13})} = \mbox{in({\tt S5})}$$
but we have not computed in({\tt S5}) yet, so this will not work!\\

If in({\tt S1}),\ldots,in({\tt S13}) are known, then we can unravel the code as before
and obtain the sets in({\tt S1}),\ldots,in({\tt S13}) once again.\\

But this means that unraveling is a function:
$$ f: {\cal P}(R)^{13} \rightarrow {\cal P}(R)^{13} $$
where $R = \{\mbox{\tt R1},\mbox{\tt R2},\ldots,\mbox{\tt R5}\}$.
A solution is a fixed point, and we want the minimal one.
\vfil
\end{slide*}
 
\begin{slide*}
Two fundamental observations:
\begin{itemize}
\item the set $D = {\cal P}(R)^{13}$ is a finite {\em lattice}:
      $$ \forall x,y\in D: x\sqcap y \in D \;\wedge\; x\sqcup y\in D$$
      where $\sqsubseteq$ is point-wise set inclusion; and
\item the unraveling function $f$ is {\em monotonic}:
      $$ \forall x,y\in D: x\sqsubseteq y \Rightarrow f(x) \sqsubseteq f(y)$$
      since $g(x) = A \cup (x - B)$ is monotonic.
\end{itemize}
\vspace*{2ex}

The fixed point theorem:

Any monotonic function $f$ on a finite lattice $D$ has the unique minimal fixed point:
$$ \bigsqcup_i f^i(\bot) $$
which is always obtained after finitely many iterations.
\vfil
\end{slide*}
 
\begin{slide*}
For $D = {\cal P}(R)^{13}$ we have that:
$$ \bot = (\emptyset,\emptyset,\ldots,\emptyset) $$
so we start with the sets in({\tt S$_i$}) = \mbox{\tt \{\}} and keep unraveling
until they no longer change. \\

Note that:
$$ \top = (R,R,\ldots,R) $$
is always a safe answer, but clearly useless and
pessimistic.

Observe that the maximal fixed-point:
$$ \sqcap_i f^i(\top) $$
may in general be smaller than $\top$.
\vfil
\end{slide*}


\begin{slide*}
Computing the minimal fixed point:\\

\renewcommand{\arraystretch}{1}
\renewcommand{\tabcolsep}{1.0ex}
\begin{scriptsize}
\begin{tt}
\begin{tabular}{|l|l|l|l|l|l|l|}
\hline
 & {\rm uses} & {\rm defs} & {\rm succ} & $\bot$ & $f(\bot)$ & $f^2(\bot)$  \\\hline
S1 &      & R1& S2& \{\}& \{\}&  \{\}\\\hline
S2 &      & R2& S3& \{\}& \{\}& \{\} \\\hline
S3 &      & R3& S4& \{\}& \{\}& \{\} \\\hline
S4 &      & R4& S5& \{\}& \{\}& \{R1\} \\\hline
S5 & R1   & R5& S6& \{\}&\{R1\} & \{R1\}  \\\hline
S6 & R5   & & S7& \{\}& \{R5\}& \{R5\}\\\hline
S7 &      & & S8,S9& \{\}& \{\}& \{R1,R2,R3\} \\\hline
S8 & R1,R2& R2& S10& \{\}& \{R1,R2\}& \{R1,R2,R4\} \\\hline
S9 & R1,R3& R3& S10& \{\}& \{R1,R3\}& \{R1,R3,R4\} \\\hline
S10 & R1,R4& R4& S11& \{\}&  \{R1,R4\}& \{R1,R4\}\\\hline
S11 & R1& R1& S12& \{\}& \{R1\} & \{R1\}\\\hline
S12 & R1& & S13& \{\}&  \{R1\}& \{R1\}\\\hline
S13 &   & & S5& \{\}&  \{\}& \{R1\}\\\hline
\end{tabular}
\end{tt}
\end{scriptsize}
\vspace*{2ex}

The function is:
$$ f(X_1,X_2,\ldots,X_{13}) = (Y_1,Y_2,\ldots,Y_{13})$$
where:
$$ Y_i = \mbox{uses}(\mbox{\tt S}_i) \cup
         ( \bigcup_{\displaystyle \mbox{\tt S}_j\!\in\!\mbox{succ}(\mbox{\tt S}_i)}
           \!\!\!\!\!X_j - \mbox{defs}(\mbox{\tt S}_i) )$$
\vfil
\end{slide*}

\begin{slide*}
\renewcommand{\arraystretch}{1}
\renewcommand{\tabcolsep}{1.0ex}
\begin{scriptsize}
\begin{tt}
\begin{tabular}{|l|l|l|l|}
\hline
& $f^3(\bot)$ & $f^4(\bot)$ & $f^5(\bot)$ \\\hline
S1 & \{\}& \{\}& \{\}\\\hline
S2 & \{\}& \{R1\}& \{R1\}\\\hline
S3 & \{R1\}& \{R1\}& \{R1\}\\\hline
S4 & \{R1\}& \{R1\}& \{R1,R2,R3\}\\\hline
S5 & \{R1\}& \{R1,R2,R3\}& \{R1,R2,R3,R4\}\\\hline
S6 & \{R1,R2,R3,R5\}& \{R1,R2,R3,R4,R5\}& \{R1,R2,R3,R4,R5\}\\\hline
S7 & \{R1,R2,R3,R4\}& \{R1,R2,R3,R4\}& \{R1,R2,R3,R4\}\\\hline
S8 & \{R1,R2,R4\}& \{R1,R2,R4\}& \{R1,R2,R4\}\\\hline
S9 & \{R1,R3,R4\}& \{R1,R3,R4\}& \{R1,R3,R4\}\\\hline
S10 & \{R1,R4\}& \{R1,R4\}& \{R1,R4\}\\\hline
S11 & \{R1\}& \{R1\}& \{R1\}\\\hline
S12 & \{R1\}& \{R1\}& \{R1,R2,R3\}\\\hline
S13 & \{R1\}& \{R1,R2,R3\}& \{R1,R2,R3,R4\}\\\hline
\multicolumn{4}{l}{}\\
%\end{tabular}
%\end{tt}
%\end{scriptsize}
%
%\begin{scriptsize}
%\begin{tt}
%\begin{tabular}{|l|l|l|l|}
\hline
& $f^6(\bot)$ & $f^7(\bot)$ & $f^8(\bot)$ \\\hline
S1 & \{\}& \{\}& \{\}\\\hline
S2 & \{R1\}& \{R1\}& \{R1\}\\\hline
S3 & \{R1,R2\}& \{R1,R2\}& \{R1,R2\}\\\hline
S4 & \{R1,R2,R3\}& \{R1,R2,R3\}& \{R1,R2,R3\}\\\hline
S5 & \{R1,R2,R3,R4\}& \{R1,R2,R3,R4\}& \{R1,R2,R3,R4\}\\\hline
S6 & \{R1,R2,R3,R4,R5\}& \{R1,R2,R3,R4,R5\}& \{R1,R2,R3,R4,R5\}\\\hline
S7 & \{R1,R2,R3,R4\}& \{R1,R2,R3,R4\}& \{R1,R2,R3,R4\}\\\hline
S8 & \{R1,R2,R4\}& \{R1,R2,R4\}& \{R1,R2,R3,R4\}\\\hline
S9 & \{R1,R3,R4\}& \{R1,R3,R4\}& \{R1,R2,R3,R4\}\\\hline
S10 & \{R1,R4\}& \{R1,R2,R3,R4\}& \{R1,R2,R3,R4\}\\\hline
S11 & \{R1,R2,R3\}& \{R1,R2,R3,R4\}& \{R1,R2,R3,R4\}\\\hline
S12 & \{R1,R2,R3,R4\}& \{R1,R2,R3,R4\}& \{R1,R2,R3,R4\}\\\hline
S13 & \{R1,R2,R3,R4\}& \{R1,R2,R3,R4\}& \{R1,R2,R3,R4\}\\\hline
\end{tabular}
\end{tt}
\end{scriptsize}
\vfil
\end{slide*}

\begin{slide*}
A turbo fixed point technique:\\

\begin{center}
\setlength{\unitlength}{3000sp}%
%
\begingroup\makeatletter\ifx\SetFigFont\undefined%
\gdef\SetFigFont#1#2#3#4#5{%
  \reset@font\fontsize{#1}{#2pt}%
  \fontfamily{#3}\fontseries{#4}\fontshape{#5}%
  \selectfont}%
\fi\endgroup%
\begin{picture}(3624,2124)(2089,-3373)
\thicklines
\put(4801,-1261){\line( 0,-1){2100}}
\put(5401,-1261){\line( 0,-1){2100}}
\put(4201,-1561){\line( 1, 0){1500}}
\put(4201,-1861){\line( 1, 0){1500}}
\put(4201,-2161){\line( 1, 0){1500}}
\put(4201,-2461){\line( 1, 0){1500}}
\put(4201,-2761){\line( 1, 0){1500}}
\put(4201,-3061){\line( 1, 0){1500}}
\put(5101,-2011){\vector(-1, 0){600}}
\put(5101,-2011){\vector(-2, 1){600}}
\put(5101,-2011){\vector(-1, 1){600}}
\put(5101,-2011){\vector(-3,-4){225}}
\put(5101,-2011){\vector( 0,-1){975}}
\put(5101,-2011){\vector(-1,-3){225}}
\put(2701,-1261){\line( 0,-1){2100}}
\put(3301,-1261){\line( 0,-1){2100}}
\put(2101,-1561){\line( 1, 0){1500}}
\put(2101,-1861){\line( 1, 0){1500}}
\put(2101,-2161){\line( 1, 0){1500}}
\put(2101,-2461){\line( 1, 0){1500}}
\put(2101,-2761){\line( 1, 0){1500}}
\put(2101,-3061){\line( 1, 0){1500}}
\put(3001,-2011){\vector(-1, 1){600}}
\put(3001,-2011){\vector(-2, 1){600}}
\put(3001,-2011){\vector(-1, 0){600}}
\put(3001,-2011){\vector(-2,-1){600}}
\put(3001,-2011){\vector(-1,-1){600}}
\put(3001,-2011){\vector(-2,-3){600}}
\end{picture}
\end{center}

The improved function is:
$$ f_{\Delta}(X_1,X_2,\ldots,X_{13}) = (Y_1,Y_2,\ldots,Y_{13})$$
where:
$$ 
Y_i = \mbox{uses}(\mbox{\tt S}_i) \cup
         ( \bigcup_{\displaystyle \mbox{\tt S}_j\!\in\!\mbox{succ}(\mbox{\tt S}_i)}
           \!\!\!\!\!Z_j - \mbox{defs}(\mbox{\tt S}_i) )
\;\;\;\;\;
$$
$$
Z_j = \left\{
\begin{array}{ll}
Y_j & \mbox{if $j>i$}\\
X_j & \mbox{otherwise}
\end{array} \right.\;\;\;\;\;\;\;\;\;\;\;\;\;\;\;\;\;\;\;\;\;\;\;\;\;\;\;\;\;\;\;\;\;
$$
\vfil
\end{slide*}

\begin{slide*}
Improved fixed point computation:\\

\renewcommand{\tabcolsep}{1.0ex}
\begin{scriptsize}
\begin{tt}
\begin{tabular}{|l|l|l|l|l|l|l|}
\hline
 & $\bot$ & $f_{\Delta}(\bot)$ & $f_{\Delta}^2(\bot)$ & $f_{\Delta}^3(\bot)$ 
\\\hline
S1 & \{\}& \{\}&  \{\} &\{\} \\\hline
S2 & \{\}& \{\}& \{R1\}  &\{R1\} \\\hline
S3 & \{\}& \{\}& \{R1,R2\}  &\{R1,R2\} \\\hline
S4 & \{\}& \{\}& \{R1,R2,R3\}  &\{R1,R2,R3\} \\\hline
S5 & \{\}&\{R1\} & \{R1,R2,R3,R4\}   &\{R1,R2,R3,R4\} \\\hline
S6 & \{\}& \{R5\}& \{R1,R2,R3,R4,R5\} &\{R1,R2,R3,R4,R5\} \\\hline
S7 & \{\}& \{\}& \{R1,R2,R3,R4\}  &\{R1,R2,R3,R4\} \\\hline
S8 & \{\}& \{R1,R2\}& \{R1,R2,R4\}  &\{R1,R2,R3,R4\} \\\hline
S9 & \{\}& \{R1,R3\}& \{R1,R3,R4\}  &\{R1,R2,R3,R4\} \\\hline
S10 & \{\}&  \{R1,R4\}& \{R1,R4\} &\{R1,R2,R3,R4\} \\\hline
S11 & \{\}& \{R1\} & \{R1\} &\{R1,R2,R3,R4\} \\\hline
S12  & \{\}&  \{R1\}& \{R1\} &\{R1,R2,R3,R4\} \\\hline
S13  & \{\}&  \{\}& \{R1\} &\{R1,R2,R3,R4\} \\\hline
\end{tabular}
\end{tt}
\end{scriptsize}
\vspace*{2ex}

Number of iterations is down from 8 to 3.

\vfil
\end{slide*}

\begin{slide*}
Liveness analysis is used for register allocation in optimizing
compilers.

In the basic block case, reduce spills to those variables that are
only in registers {\em and\/} live.

In the intraprocedural case, construct a graph whose nodes are
variables:
\begin{center}
\setlength{\unitlength}{2000sp}%
%
\begingroup\makeatletter\ifx\SetFigFont\undefined%
\gdef\SetFigFont#1#2#3#4#5{%
  \reset@font\fontsize{#1}{#2pt}%
  \fontfamily{#3}\fontseries{#4}\fontshape{#5}%
  \selectfont}%
\fi\endgroup%
\begin{picture}(4741,2491)(2093,-4419)
\thinlines
\put(3301,-2461){\circle{450}}
\put(4951,-2161){\circle{450}}
\put(3301,-4186){\circle{450}}
\put(5851,-3811){\circle{450}}
\put(6601,-2161){\circle{450}}
\put(2326,-3286){\circle{450}}
\put(3526,-2461){\line( 4, 1){1200}}
\put(3301,-2686){\line( 0,-1){1275}}
\put(4951,-2386){\line( 3,-4){900}}
\put(3076,-2461){\line(-5,-4){750}}
\put(3076,-4186){\line(-1, 1){675}}
\put(2251,-3361){\makebox(0,0)[lb]{\smash{\SetFigFont{8}{14.4}{\ttdefault}{\mddefault}{\updefault}a}}}
\put(3226,-4261){\makebox(0,0)[lb]{\smash{\SetFigFont{8}{14.4}{\ttdefault}{\mddefault}{\updefault}c}}}
\put(6526,-2236){\makebox(0,0)[lb]{\smash{\SetFigFont{8}{14.4}{\ttdefault}{\mddefault}{\updefault}f}}}
\put(3226,-2536){\makebox(0,0)[lb]{\smash{\SetFigFont{8}{14.4}{\ttdefault}{\mddefault}{\updefault}b}}}
\put(4876,-2236){\makebox(0,0)[lb]{\smash{\SetFigFont{8}{14.4}{\ttdefault}{\mddefault}{\updefault}d}}}
\put(5776,-3886){\makebox(0,0)[lb]{\smash{\SetFigFont{8}{14.4}{\ttdefault}{\mddefault}{\updefault}e}}}
\end{picture}
\end{center}
and where edges connect nodes that are live at the same time.

Register allocation is now reduced to finding a minimal graph coloring:
\begin{verbatim}
   { {a,d,f}, {b,e}, {c} }
\end{verbatim}
and assigning a register to each color.

\vfil
\end{slide*}

\begin{slide*}
Liveness analysis is a {\em backwards\/} analysis, since we
unravel from the future towards the past.\\

An example of a {\em forwards\/} analysis is constant propagation:\\


\setlength{\unitlength}{2000sp}%
%
\begingroup\makeatletter\ifx\SetFigFont\undefined%
\gdef\SetFigFont#1#2#3#4#5{%
  \reset@font\fontsize{#1}{#2pt}%
  \fontfamily{#3}\fontseries{#4}\fontshape{#5}%
  \selectfont}%
\fi\endgroup%
\begin{picture}(3000,3765)(2401,-5086)
\thinlines
\put(3301,-1561){\vector( 0,-1){600}}
\put(3301,-2461){\vector( 0,-1){600}}
\put(3301,-3361){\vector( 0,-1){600}}
\put(3301,-4261){\vector( 0,-1){600}}
\put(2401,-1486){\makebox(0,0)[lb]{\smash{\SetFigFont{8}{14.4}{\ttdefault}{\mddefault}{\updefault}S1:  mov 3,R1}}}
\put(2401,-2386){\makebox(0,0)[lb]{\smash{\SetFigFont{8}{14.4}{\ttdefault}{\mddefault}{\updefault}S2:  mov 4,R2}}}
\put(2401,-3286){\makebox(0,0)[lb]{\smash{\SetFigFont{8}{14.4}{\ttdefault}{\mddefault}{\updefault}S3:  add R1,R2,R3}}}
\put(2401,-4186){\makebox(0,0)[lb]{\smash{\SetFigFont{8}{14.4}{\ttdefault}{\mddefault}{\updefault}S4:  mov R3,R0}}}
\put(2401,-5086){\makebox(0,0)[lb]{\smash{\SetFigFont{8}{14.4}{\ttdefault}{\mddefault}{\updefault}S5:  return}}}
\put(5401,-1486){\makebox(0,0)[lb]{\smash{\SetFigFont{8}{14.4}{\ttdefault}{\mddefault}{\updefault}\{(R0,?),(R1,?),(R2,?),(R3,?)\}}}}
\put(5401,-2386){\makebox(0,0)[lb]{\smash{\SetFigFont{8}{14.4}{\ttdefault}{\mddefault}{\updefault}\{(R0,?),(R1,3),(R2,?),(R3,?)\}}}}
\put(5401,-3286){\makebox(0,0)[lb]{\smash{\SetFigFont{8}{14.4}{\ttdefault}{\mddefault}{\updefault}\{(R0,?),(R1,3),(R2,4),(R3,?)\}}}}
\put(5401,-4186){\makebox(0,0)[lb]{\smash{\SetFigFont{8}{14.4}{\ttdefault}{\mddefault}{\updefault}\{(R0,?),(R1,3),(R2,4),(R3,7)\}}}}
\put(5401,-5086){\makebox(0,0)[lb]{\smash{\SetFigFont{8}{14.4}{\ttdefault}{\mddefault}{\updefault}\{(R0,7),(R1,3),(R2,4),(R3,7)\}}}}
\end{picture}
\vfil
\end{slide*}

\begin{slide*}
A basic static analysis of JOOS and other object-oriented languages
is {\em type inference}.\\

Given an expression, what are the possible classes of the
objects to which it may evaluate?\\

The exact answer is undecidable, so we must conservatively approximate:
\begin{itemize}
\item we will accept a set that is too large;
\item we want it as small as possible; and
\item a trivial answer includes all classes.
\end{itemize}
\vspace*{2ex}

This analysis is interprocedural and requires access to the whole program.
\vfil
\end{slide*}

\begin{slide*}
Possible uses of type inference:\\

\begin{itemize}
\item inline methods when there is only one possible receiver;
\item eliminate run-time checks that can be decided statically;
\item remove code that is never executed; and
\item approximate the control flow graph to enable other
static analyses.
\end{itemize}
\vspace*{2ex}

In each case, smaller inferred sets will give better results.
\vfil
\end{slide*}

\begin{slide*}
The constraint technique:\\

\begin{itemize}
\item assign a variable \CC{E} to each occurrence of an expression {\tt E};
\item assign a variable \CC{m} to each occurrence of a method {\tt m};
\item the variables range over the set of all classes 
$C = \{\mbox{\tt C}_1,\mbox{\tt C}_2,\ldots,\mbox{\tt C}_n\}$;
\item each parse tree node generates a local constraint on the variables; and
\item the global minimal solution of these constraints is finally computed.
\end{itemize}
\vspace*{2ex}

Again, we must compute a minimal fixed point in a finite lattice.
\vfil
\end{slide*}

\begin{slide*}
Each constraint models the flow of objects:
\begin{itemize}
\item the assignment ``{\tt i = E}'' yields: $\CC{E} \subseteq \CC{i}$;
\item the creation ``{\tt new C()}'' yields: $\{\mbox{\tt C}\} \subseteq \CC{new C()}$;
\item the cast ``{\tt C(E)}'' yields:\\ $\{\mbox{\tt C}\} \subseteq \CC{C(E)}$;
\item the constant ``{\tt this}'' yields: $\{\mbox{\tt C}\} \subseteq \CC{this}$,\\
where {\tt C} is the surrounding class; and
\item the statement ``{\tt return E}'' yields: $\CC{E} \subseteq \CC{m}$,\\ 
where {\tt m} is the surrounding method.
\end{itemize}
\vfil
\end{slide*}

\begin{slide*}
The method invocation:

{\tt E.m(E$_1$,E$_2$,\ldots,E$_k$)}

yields the {\em conditional\/} constraints:
$$
\mbox{\tt C}_i \in \CC{E} \Rightarrow \left\{
\begin{array}{l}
\CC{E$_1$} \subseteq \CC{x$_1$}\\
\CC{E$_2$} \subseteq \CC{x$_2$}\\
\vdots\\
\CC{E$_k$} \subseteq \CC{x$_k$}\\
\CC{m} \subseteq \CC{E.m(E$_1$,E$_2$,\ldots,E$_k$)}
\end{array}\right.
$$
whenever the class \mbox{\tt C}$_i$ implements a method named {\tt m} which accepts $k$
arguments named {\tt x}$_1$, {\tt x}$_2$, \ldots, {\tt x}$_k$.
\vfil
\end{slide*}

\begin{slide*}
Since the constraint:
$$v\subseteq w$$
holds if and only if the equality:
$$w = v\cup w$$
does, we can rewrite a set of constraints into a function:
$$ f: {\cal P}(C)^k \rightarrow {\cal P}(C)^k $$
such that fixed-points of $f$ correspond to solutions
to the constraints.
\vfil
\end{slide*}

\begin{slide*}
For the example constraints:
$$\begin{array}{l}
v_1 \subseteq v_2\\
\mbox{\tt C}_3 \in v_2 \Rightarrow v_3 \subseteq v_1\\
\{\mbox{C}_7\} \subseteq v_3
\end{array}
$$
we get the function:
$$
\begin{array}{l}
f(X_1,X_2,X_3) = \\
\;\left\{\!\!\begin{array}{ll}
(X_1\!\cup\!X_3, X_1\!\cup\!X_2, \{\mbox{C}_7\}\!\cup\!X_3) & \!\mbox{if~} \mbox{\tt C}_3\!\in\!X_2\\
(X_1,X_1\!\cup\!X_2, \{\mbox{C}_7\}\!\cup\!X_3) & \!\mbox{otherwise}
\end{array}\right.
\end{array}
$$
\vfil
\end{slide*}

\begin{slide*}
Solving the constraints:
\begin{itemize}
\item ${\cal P}(C)^k$ is a finite lattice;
\item each function $f$ is monotonic; and
\item the least fixed point of $f$ is the unique smallest solution of the constraints.
\end{itemize}
\vspace*{2ex}

\begin{center}
\setlength{\unitlength}{2200sp}%
%
\begingroup\makeatletter\ifx\SetFigFont\undefined%
\gdef\SetFigFont#1#2#3#4#5{%
  \reset@font\fontsize{#1}{#2pt}%
  \fontfamily{#3}\fontseries{#4}\fontshape{#5}%
  \selectfont}%
\fi\endgroup%
\begin{picture}(5424,5805)(1489,-6886)
\thinlines
\put(3601,-2761){\circle*{150}}
\put(4426,-3361){\circle*{150}}
\put(4201,-1261){\line(-1,-1){2700}}
\put(1501,-3961){\line( 1,-1){2700}}
\put(4201,-6661){\line( 1, 1){2700}}
\put(6901,-3961){\line(-1, 1){2700}}
\put(3601,-3361){\vector( 0, 1){525}}
\put(3601,-4261){\vector( 2, 1){600}}
\put(3901,-4861){\vector(-1, 2){300}}
\put(3601,-5161){\vector( 1, 1){300}}
\put(4201,-5461){\vector(-2, 1){600}}
\put(4501,-6061){\vector(-1, 2){300}}
\put(4201,-6661){\vector( 1, 2){300}}
\put(4201,-3961){\vector(-1, 1){600}}
\put(3676,-2836){\line( 3,-2){675}}
\put(4101,-1126){\makebox(0,0)[lb]{\smash{\SetFigFont{8}{14.4}{\familydefault}{\mddefault}{\updefault}$\top$}}}
\put(4101,-6920){\makebox(0,0)[lb]{\smash{\SetFigFont{8}{14.4}{\familydefault}{\mddefault}{\updefault}$\bot$}}}
\put(3751,-2816){\makebox(0,0)[lb]{\smash{\SetFigFont{8}{14.4}{\familydefault}{\mddefault}{\updefault}fixed point}}}
\put(4576,-3436){\makebox(0,0)[lb]{\smash{\SetFigFont{8}{14.4}{\familydefault}{\mddefault}{\updefault}ideal answer}}}
\end{picture}
\end{center}
\vfil
\end{slide*}

\begin{slide*}
A tiny JOOS sketch:
\begin{scriptsize}
\begin{verbatim}
public class A {
  public A() { super(); }
  public A id(A x) { return x; }
}

public class B extends A {
  public B() { super(); }
  public B me() { return (B)(new A()).id(this); }
}
\end{verbatim}
\end{scriptsize}

The generated constraints are:
\begin{scriptsize}
\begin{tabbing}
$\CC{x}_{\mbox{\tt A}} \subseteq \CC{id}_{\mbox{\tt A}}$\\
$\CC{x}_{\mbox{\tt B}} \subseteq \CC{id}_{\mbox{\tt B}}$\\
$\CC{(B)(new A()).id(this)} \subseteq \CC{me}$\\
$\{\mbox{\tt B}\} \subseteq \CC{(B)(new A()).id(this)}$\\
$\{\mbox{\tt B}\} \subseteq \CC{this}$\\
$\{\mbox{\tt A}\} \subseteq \CC{new A()}$\\
$\mbox{\tt A} \in \CC{new A()} \Rightarrow \CC{this} \subseteq \CC{x}_{\mbox{\tt A}}$\\
$\mbox{\tt A} \in \CC{new A()} \Rightarrow \CC{id}_{\mbox{\tt A}} \subseteq \CC{(new A()).id(this)}$\\
$\mbox{\tt B} \in \CC{new A()} \Rightarrow \CC{this} \subseteq \CC{x}_{\mbox{\tt B}}$\\
$\mbox{\tt B} \in \CC{new A()} \Rightarrow \CC{id}_{\mbox{\tt B}} \subseteq \CC{(new A()).id(this)}$
\end{tabbing}
\end{scriptsize}
The minimal solution is:
\begin{scriptsize}
\begin{tabbing}
$\CC{new A()} = \{\mbox{\tt A}\}$\\
$\CC{x}_{\mbox{\tt A}} = \CC{id}_{\mbox{\tt A}} = \CC{this} = \CC{(new A()).id(this)} = \{\mbox{\tt B}\}$\\
$\CC{(B)(new A()).id(this)} = \{\mbox{\tt B}\}$\\
$\CC{x}_{\mbox{\tt B}} = \CC{id}_{\mbox{\tt B}} = \{\}$
\end{tabbing}
\end{scriptsize}
\vfil
\end{slide*}

\begin{slide*}
The generated code for the {\tt me} method is:
\begin{scriptsize}
\begin{verbatim}
.method public me()LB;
  .limit locals 1
  .limit stack 2
  new A
  dup
  invokenonvirtual A/<init>()V
  aload_0
  invokevirtual A/id(LA;)LA;
  checkcast B
  areturn
.end method
\end{verbatim}
\end{scriptsize}

The information $\CC{new A().id(this)} = \{\mbox{\tt B}\}$ 
eliminates the {\tt checkcast} instruction.

That $\CC{new A()} = \{\mbox{\tt A}\}$ is a singleton further 
allows inlining of the {\tt id} method:

\begin{scriptsize}
\begin{verbatim}
.method public me()LB;
  .limit locals 1
  .limit stack 1
  aload_0
  areturn
.end method
\end{verbatim}
\end{scriptsize}

With type inference, many little methods become almost free.
\vfil
\end{slide*}

\begin{slide*}
Improving analyses by transformations:
\begin{itemize}
\item let $P$ be our set of programs;
\item let $S: P \rightarrow D$ be an ideal static analysis (uncomputable); and
\item let $T: P \rightarrow P$ be a program transformation that preserves the semantics.
\end{itemize}
Since $S$ gives the ideal information, clearly $S(T(p)) = S(p)$ for all $p\in P$.\\

However, if $A: P \rightarrow D$ is a conservative approximation to $S$, then
$A(T(p))$ may be different from $A(p)$, perhaps even better.
\vfil
\end{slide*}

\begin{slide*}
Transformations boost analyses: \\

~~~~~~~~~\setlength{\unitlength}{2300sp}%
%
\begingroup\makeatletter\ifx\SetFigFont\undefined%
\gdef\SetFigFont#1#2#3#4#5{%
  \reset@font\fontsize{#1}{#2pt}%
  \fontfamily{#3}\fontseries{#4}\fontshape{#5}%
  \selectfont}%
\fi\endgroup%
\begin{picture}(4824,5805)(1489,-5386)
\thinlines
\put(1576,-1036){\line( 1, 0){ 75}}
\put(1726,-1036){\line( 1, 0){225}}
\put(1576,-1111){\line( 1, 0){225}}
\put(1876,-1111){\line( 1, 0){150}}
\put(1576,-1186){\line( 1, 0){150}}
\put(1801,-1186){\line( 1, 0){225}}
\put(1576,-1261){\line( 1, 0){ 75}}
\put(1726,-1261){\line( 1, 0){ 75}}
\put(1876,-1261){\line( 1, 0){150}}
\put(1576,-1336){\line( 1, 0){ 75}}
\put(1726,-1336){\line( 1, 0){225}}
\put(1576,-1411){\line( 1, 0){225}}
\put(1876,-1411){\line( 1, 0){150}}
\put(1576,-1486){\line( 1, 0){150}}
\put(1801,-1486){\line( 1, 0){225}}
\put(1576,-1561){\line( 1, 0){ 75}}
\put(1726,-1561){\line( 1, 0){ 75}}
\put(1876,-1561){\line( 1, 0){150}}
\put(1576,-1636){\line( 1, 0){ 75}}
\put(1726,-1636){\line( 1, 0){225}}
\put(1576,-1711){\line( 1, 0){225}}
\put(1876,-1711){\line( 1, 0){150}}
\put(1576,-1786){\line( 1, 0){150}}
\put(1801,-1786){\line( 1, 0){225}}
\put(1576,-1861){\line( 1, 0){ 75}}
\put(1726,-1861){\line( 1, 0){ 75}}
\put(1876,-1861){\line( 1, 0){150}}
\put(1576,-2836){\line( 1, 0){ 75}}
\put(1726,-2836){\line( 1, 0){225}}
\put(1576,-2911){\line( 1, 0){225}}
\put(1876,-2911){\line( 1, 0){150}}
\put(1576,-2986){\line( 1, 0){150}}
\put(1801,-2986){\line( 1, 0){225}}
\put(1576,-3061){\line( 1, 0){ 75}}
\put(1726,-3061){\line( 1, 0){ 75}}
\put(1876,-3061){\line( 1, 0){150}}
\put(1576,-3136){\line( 1, 0){ 75}}
\put(1726,-3136){\line( 1, 0){225}}
\put(1576,-3211){\line( 1, 0){225}}
\put(1876,-3211){\line( 1, 0){150}}
\put(1576,-3286){\line( 1, 0){150}}
\put(1801,-3286){\line( 1, 0){225}}
\put(1576,-3361){\line( 1, 0){ 75}}
\put(1726,-3361){\line( 1, 0){ 75}}
\put(1876,-3361){\line( 1, 0){150}}
\put(1576,-4561){\line( 1, 0){ 75}}
\put(1726,-4561){\line( 1, 0){225}}
\put(1576,-4636){\line( 1, 0){225}}
\put(1876,-4636){\line( 1, 0){150}}
\put(1576,-4711){\line( 1, 0){150}}
\put(1801,-4711){\line( 1, 0){225}}
\put(1576,-4786){\line( 1, 0){ 75}}
\put(1726,-4786){\line( 1, 0){ 75}}
\put(1876,-4786){\line( 1, 0){150}}
\put(1801,-2761){\vector( 0, 1){600}}
\put(1576,-1936){\line( 1, 0){300}}
\put(1576,-2011){\line( 1, 0){ 75}}
\put(1726,-2011){\line( 1, 0){300}}
\put(1576,-2086){\line( 1, 0){150}}
\put(1801,-2086){\line( 1, 0){ 75}}
\put(1951,-2086){\line( 1, 0){ 75}}
\put(1576,-3361){\line( 1, 0){ 75}}
\put(1726,-3361){\line( 1, 0){225}}
\put(1576,-3436){\line( 1, 0){225}}
\put(1876,-3436){\line( 1, 0){150}}
\put(1576,-3511){\line( 1, 0){150}}
\put(1801,-3511){\line( 1, 0){225}}
\put(1576,-3586){\line( 1, 0){ 75}}
\put(1726,-3586){\line( 1, 0){ 75}}
\put(1876,-3586){\line( 1, 0){150}}
\put(1576,-4336){\line( 1, 0){ 75}}
\put(1726,-4336){\line( 1, 0){225}}
\put(1576,-4411){\line( 1, 0){225}}
\put(1876,-4411){\line( 1, 0){150}}
\put(1576,-4486){\line( 1, 0){150}}
\put(1801,-4486){\line( 1, 0){225}}
\put(1576,-4561){\line( 1, 0){ 75}}
\put(1726,-4561){\line( 1, 0){ 75}}
\put(1876,-4561){\line( 1, 0){150}}
\put(4351,-3061){\circle*{120}}
\put(3976,-811){\circle*{120}}
\put(4576,-1411){\circle*{120}}
\put(4351,-2311){\circle*{120}}
\put(1501,-4861){\framebox(600,600){}}
\put(1501,-3661){\framebox(600,900){}}
\put(1501,-2161){\framebox(600,1200){}}
\put(1801,-4261){\vector( 0, 1){600}}
\put(2101,-4561){\vector( 3, 2){2250}}
\put(2101,-1561){\vector( 3,-2){2250}}
\put(2101,-3061){\vector( 1, 0){2175}}
\put(2101,-1561){\vector( 3,-1){2250}}
\put(4351,-3061){\line( 0, 1){750}}
\put(2101,-3061){\vector( 3, 2){2475}}
\put(4351,-2311){\line( 1, 4){225}}
\put(4576,-1411){\line(-1, 1){600}}
\put(2101,-4561){\vector( 1, 2){1875}}
\put(4501,239){\line(-2,-3){1800}}
\put(2701,-2461){\line( 2,-3){1800}}
\put(4501,-5161){\line( 2, 3){1800}}
\put(6301,-2461){\line(-2, 3){1800}}
\put(4400,314){\makebox(0,0)[lb]{\smash{\SetFigFont{8}{14.4}{\familydefault}{\mddefault}{\updefault}$\top$}}}
\put(4390,-5410){\makebox(0,0)[lb]{\smash{\SetFigFont{8}{14.4}{\familydefault}{\mddefault}{\updefault}$\bot$}}}
\put(1576,-2536){\makebox(0,0)[lb]{\smash{\SetFigFont{8}{14.4}{\familydefault}{\mddefault}{\updefault}$T$}}}
\put(1576,-4036){\makebox(0,0)[lb]{\smash{\SetFigFont{8}{14.4}{\familydefault}{\mddefault}{\updefault}$T$}}}
\put(2091,-4036){\makebox(0,0)[lb]{\smash{\SetFigFont{8}{14.4}{\familydefault}{\mddefault}{\updefault}$A$}}}
\put(2446,-4531){\makebox(0,0)[lb]{\smash{\SetFigFont{8}{14.4}{\familydefault}{\mddefault}{\updefault}$S$}}}
\put(2326,-3256){\makebox(0,0)[lb]{\smash{\SetFigFont{8}{14.4}{\familydefault}{\mddefault}{\updefault}$S$}}}
\put(2181,-2776){\makebox(0,0)[lb]{\smash{\SetFigFont{8}{14.4}{\familydefault}{\mddefault}{\updefault}$A$}}}
\put(2271,-2041){\makebox(0,0)[lb]{\smash{\SetFigFont{8}{14.4}{\familydefault}{\mddefault}{\updefault}$S$}}}
\put(2551,-1651){\makebox(0,0)[lb]{\smash{\SetFigFont{8}{14.4}{\familydefault}{\mddefault}{\updefault}$A$}}}
\end{picture}\\
The transformation $T$:
\begin{itemize}
\item may unfold the program to make it more explicit; or
\item may itself be an optimization.
\end{itemize}
\vfil
\end{slide*}

\end{document}

